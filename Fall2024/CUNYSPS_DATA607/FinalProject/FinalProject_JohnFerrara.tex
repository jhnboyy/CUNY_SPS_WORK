% Options for packages loaded elsewhere
\PassOptionsToPackage{unicode}{hyperref}
\PassOptionsToPackage{hyphens}{url}
\documentclass[
]{article}
\usepackage{xcolor}
\usepackage[margin=1in]{geometry}
\usepackage{amsmath,amssymb}
\setcounter{secnumdepth}{-\maxdimen} % remove section numbering
\usepackage{iftex}
\ifPDFTeX
  \usepackage[T1]{fontenc}
  \usepackage[utf8]{inputenc}
  \usepackage{textcomp} % provide euro and other symbols
\else % if luatex or xetex
  \usepackage{unicode-math} % this also loads fontspec
  \defaultfontfeatures{Scale=MatchLowercase}
  \defaultfontfeatures[\rmfamily]{Ligatures=TeX,Scale=1}
\fi
\usepackage{lmodern}
\ifPDFTeX\else
  % xetex/luatex font selection
\fi
% Use upquote if available, for straight quotes in verbatim environments
\IfFileExists{upquote.sty}{\usepackage{upquote}}{}
\IfFileExists{microtype.sty}{% use microtype if available
  \usepackage[]{microtype}
  \UseMicrotypeSet[protrusion]{basicmath} % disable protrusion for tt fonts
}{}
\makeatletter
\@ifundefined{KOMAClassName}{% if non-KOMA class
  \IfFileExists{parskip.sty}{%
    \usepackage{parskip}
  }{% else
    \setlength{\parindent}{0pt}
    \setlength{\parskip}{6pt plus 2pt minus 1pt}}
}{% if KOMA class
  \KOMAoptions{parskip=half}}
\makeatother
\usepackage{color}
\usepackage{fancyvrb}
\newcommand{\VerbBar}{|}
\newcommand{\VERB}{\Verb[commandchars=\\\{\}]}
\DefineVerbatimEnvironment{Highlighting}{Verbatim}{commandchars=\\\{\}}
% Add ',fontsize=\small' for more characters per line
\usepackage{framed}
\definecolor{shadecolor}{RGB}{248,248,248}
\newenvironment{Shaded}{\begin{snugshade}}{\end{snugshade}}
\newcommand{\AlertTok}[1]{\textcolor[rgb]{0.94,0.16,0.16}{#1}}
\newcommand{\AnnotationTok}[1]{\textcolor[rgb]{0.56,0.35,0.01}{\textbf{\textit{#1}}}}
\newcommand{\AttributeTok}[1]{\textcolor[rgb]{0.13,0.29,0.53}{#1}}
\newcommand{\BaseNTok}[1]{\textcolor[rgb]{0.00,0.00,0.81}{#1}}
\newcommand{\BuiltInTok}[1]{#1}
\newcommand{\CharTok}[1]{\textcolor[rgb]{0.31,0.60,0.02}{#1}}
\newcommand{\CommentTok}[1]{\textcolor[rgb]{0.56,0.35,0.01}{\textit{#1}}}
\newcommand{\CommentVarTok}[1]{\textcolor[rgb]{0.56,0.35,0.01}{\textbf{\textit{#1}}}}
\newcommand{\ConstantTok}[1]{\textcolor[rgb]{0.56,0.35,0.01}{#1}}
\newcommand{\ControlFlowTok}[1]{\textcolor[rgb]{0.13,0.29,0.53}{\textbf{#1}}}
\newcommand{\DataTypeTok}[1]{\textcolor[rgb]{0.13,0.29,0.53}{#1}}
\newcommand{\DecValTok}[1]{\textcolor[rgb]{0.00,0.00,0.81}{#1}}
\newcommand{\DocumentationTok}[1]{\textcolor[rgb]{0.56,0.35,0.01}{\textbf{\textit{#1}}}}
\newcommand{\ErrorTok}[1]{\textcolor[rgb]{0.64,0.00,0.00}{\textbf{#1}}}
\newcommand{\ExtensionTok}[1]{#1}
\newcommand{\FloatTok}[1]{\textcolor[rgb]{0.00,0.00,0.81}{#1}}
\newcommand{\FunctionTok}[1]{\textcolor[rgb]{0.13,0.29,0.53}{\textbf{#1}}}
\newcommand{\ImportTok}[1]{#1}
\newcommand{\InformationTok}[1]{\textcolor[rgb]{0.56,0.35,0.01}{\textbf{\textit{#1}}}}
\newcommand{\KeywordTok}[1]{\textcolor[rgb]{0.13,0.29,0.53}{\textbf{#1}}}
\newcommand{\NormalTok}[1]{#1}
\newcommand{\OperatorTok}[1]{\textcolor[rgb]{0.81,0.36,0.00}{\textbf{#1}}}
\newcommand{\OtherTok}[1]{\textcolor[rgb]{0.56,0.35,0.01}{#1}}
\newcommand{\PreprocessorTok}[1]{\textcolor[rgb]{0.56,0.35,0.01}{\textit{#1}}}
\newcommand{\RegionMarkerTok}[1]{#1}
\newcommand{\SpecialCharTok}[1]{\textcolor[rgb]{0.81,0.36,0.00}{\textbf{#1}}}
\newcommand{\SpecialStringTok}[1]{\textcolor[rgb]{0.31,0.60,0.02}{#1}}
\newcommand{\StringTok}[1]{\textcolor[rgb]{0.31,0.60,0.02}{#1}}
\newcommand{\VariableTok}[1]{\textcolor[rgb]{0.00,0.00,0.00}{#1}}
\newcommand{\VerbatimStringTok}[1]{\textcolor[rgb]{0.31,0.60,0.02}{#1}}
\newcommand{\WarningTok}[1]{\textcolor[rgb]{0.56,0.35,0.01}{\textbf{\textit{#1}}}}
\usepackage{longtable,booktabs,array}
\usepackage{calc} % for calculating minipage widths
% Correct order of tables after \paragraph or \subparagraph
\usepackage{etoolbox}
\makeatletter
\patchcmd\longtable{\par}{\if@noskipsec\mbox{}\fi\par}{}{}
\makeatother
% Allow footnotes in longtable head/foot
\IfFileExists{footnotehyper.sty}{\usepackage{footnotehyper}}{\usepackage{footnote}}
\makesavenoteenv{longtable}
\usepackage{graphicx}
\makeatletter
\newsavebox\pandoc@box
\newcommand*\pandocbounded[1]{% scales image to fit in text height/width
  \sbox\pandoc@box{#1}%
  \Gscale@div\@tempa{\textheight}{\dimexpr\ht\pandoc@box+\dp\pandoc@box\relax}%
  \Gscale@div\@tempb{\linewidth}{\wd\pandoc@box}%
  \ifdim\@tempb\p@<\@tempa\p@\let\@tempa\@tempb\fi% select the smaller of both
  \ifdim\@tempa\p@<\p@\scalebox{\@tempa}{\usebox\pandoc@box}%
  \else\usebox{\pandoc@box}%
  \fi%
}
% Set default figure placement to htbp
\def\fps@figure{htbp}
\makeatother
\setlength{\emergencystretch}{3em} % prevent overfull lines
\providecommand{\tightlist}{%
  \setlength{\itemsep}{0pt}\setlength{\parskip}{0pt}}
\usepackage{bookmark}
\IfFileExists{xurl.sty}{\usepackage{xurl}}{} % add URL line breaks if available
\urlstyle{same}
\hypersetup{
  pdftitle={Final Project - DATA 606 \& DATA 607},
  pdfauthor={John Ferrara},
  hidelinks,
  pdfcreator={LaTeX via pandoc}}

\title{Final Project - DATA 606 \& DATA 607}
\author{John Ferrara}
\date{2024-12-09}

\begin{document}
\maketitle

\subsection{Abstract}\label{abstract}

This observational study examines whether completed housing-related
construction correlates with the adult homeless population, and how
homelessness correlates with public syringe recovery in NYC parks. Using
data sourced from NYC Open Data, two questions were analyzed: (1) Does
completed housing construction projects over the preceding five-year
window affect the size of the adult homeless population? (2) Does the
size of the adult homeless population correlate with disposed syringe
recovery rates? Data from the NYC Department of City Planning for
construction projects, NYC Parks Department syringe recovery data, and
NYC Department of Homelessness Services homeless population data were
analyzed using linear regression models, both at the community district
and borough levels.

Several statistically significant relationships were found. At the
community district level, more completed construction projects directly
correlated with a larger homeless population, possibly reflecting the
concentration of construction jobs in wealthier or gentrifying areas.
Additionally, an increase in the adult homeless population directly
correlated with higher syringe recovery rates in NYC parks. At the more
generalized borough level, the correlation between completed
construction projects and homelessness was stronger, with the
construction projects explaining roughly 67\% of the variance in
homelessness populations. Borough-level analysis also revealed an
indirect relationship between the average year-over-year increases in
completed construction projects and homelessness, suggesting targeted
development may help reduce homelessness.

These findings highlight connections between housing policy,
homelessness, and syringe recovery as a singular dimension of public
health. While limitations for this analysis exist, such as a lack of
housing data granularity and a lack of non-shelter based homeless
counts, the analysis underscores the importance of public data in
addressing complex urban challenges.

\subsection{Introduction and Overview}\label{introduction-and-overview}

As most know, in recent years the cost of housing has been outpacing
wage growth1. Rising housing costs have led to a series of policy
issues. While some impacts of high housing costs are obvious, others are
less apparent. This analysis explores whether one such issue --- the
recovery of used syringes in public parks --- is correlated to the
housing crisis. In short, the overarching question: Does housing impact
this seemingly separate public health and safety issue of public, or
improper, syringe disposal?

The overarching question for this observational study must be broken
down into two different main questions. The following are the two
sub-questions and their respective hypotheses:

\begin{itemize}
\tightlist
\item
  \emph{Question 1}: Does the Number of Housing Related Construction
  Projects for the Previous 5 Years correlate with the Size of Adult
  Homeless Population?

  \begin{itemize}
  \tightlist
  \item
    \emph{Null Hypothesis}: The number of housing related construction
    projects for the preceding 5 year window period \textbf{does not}
    correlate with the adult homeless population.
  \item
    \emph{Alternative Hypothesis}: The number of housing related
    construction projects for the preceding 5 year window period
    \textbf{does} correlated with the adult homeless population.
  \item
    \emph{Independent Variable}: The independent variable in this
    question is the completed housing-related construction projects
    completed in the preceding 5 year window from when the homeless
    population was counted.
  \item
    \emph{Dependent Variable}: The dependent variable in this question
    is the Adult homeless population.
  \end{itemize}
\item
  \emph{Question 2}: Does the size of the Adult Homeless Population
  correlate with the Number of Used Syringes recovered by NYC Parks?

  \begin{itemize}
  \tightlist
  \item
    \emph{Null Hypothesis}: The size of the Adult homeless population
    \textbf{does not} correlate with the number of used syringes
    recovered in NYC Parks.
  \item
    \emph{Alternative Hypothesis}: The size of the Adult homeless
    population \textbf{does} correlate with the number of used syringes
    recovered in NYC Parks.
  \item
    \emph{Independent Variable}: The independent variable in this
    question is the adult unhoused population.
  \item
    \emph{Dependent Variable}: The dependent variable in this question
    is the total syringes collected from NYC Parks and public safe
    disposal sites.
  \end{itemize}
\end{itemize}

It should be noted that these are large questions. Questions that are
much more nuanced and complex than any single analysis can outline, so
while this study seeks to demonstrate a correlation between varying
dimensions of public policy implications, it is by no means an attempt
at a solution. Rather, the study seeks to determine if there is an
overlap between these issues using several sources of public data.

\subsection{Data Sources}\label{data-sources}

There are multiple data sources used in this study. Most, if not all,
were sourced from \href{https://opendata.cityofnewyork.us/}{NYC Open
Data} a website that provides a multitude of public data sets generated
by New York City government agencies. The data used in this analysis
were:

\textbf{- Dataset 1:
\href{https://data.cityofnewyork.us/Public-Safety/Summary-of-Syringe-Data-in-NYC-Parks/t8xi-d5wb/about_data}{NYC
Parks Syringe Collection Data}}

\emph{Overview \& Data Assumptions}

This dataset is from the NYC Parks Department. NYC Parks department
staff, along with the staff of various community non-profit
organizations, collect used syringes discarded improperly in public
parks and log the totals. These collection totals are grouped by each
Parks District, which is an internal administrative geographic boundary
for the New York City Parks Department. Each row of data is the
equivalent of a day's total collection of syringes. Within this data set
there are three total columns that outline the number of syringes
collected. They are as follows:

\begin{itemize}
\item
  Total Kiosk Syringes Collected: This is the total number of syringes
  that are collected from the city's safe disposal kiosks.
\item
  Total Ground Syringes Collected: This is the total number of syringes
  that are collected from the ground, or just generally found to be
  improperly disposed of.
\item
  Total Syringes Collected: This is the total number of syringes from
  both the ``ground'' category and the ``kiosk'' category.
\end{itemize}

This analysis makes use of the total ground syringes collected, not the
kiosk or ground syringes. While assumptions could be made about housed
and unhoused population's syringe disposal habits, the total number of
syringes publicly disposed by either method should capture what is
needed for this analysis.

\emph{Source Format \& Ingestion Method}

Sourced from CSV formatted files. The NYC Open Data API was used for
iteratively ingesting this data from CSV structured formatting.

\textbf{- Dataset 2:
\href{https://data.cityofnewyork.us/Housing-Development/Housing-Database/6umk-irkx/about_data}{NYC
Dept. City Planning Housing Database}}

\emph{Overview \& Data Assumptions}

This data is from the NYC Department of City Planning (DCP), the data
contains information from the NYC Department of Buildings (DOB) for how
many construction and demolition jobs are within a specific
municipality-based geographies. For this analysis used at Community
Districts for the main geographic boundary. The analysis leverages
yearly counts of completed construction projects. The annual count data
does not provide nuance between the three main types of construction
projects included in the data (new buildings, major alterations, and
demolitions), however, for the sake of this analysis, a larger number of
projects is assumed to be targeted at enhancing living conditions and
making new units available to the housing market. Annual counts from
2010 through 2020 by Community District were used in varying relative
5-year windows. How this was done is explained in detail within the
methodology section.

The original download is a zipfile with multiple geographies, for this
analysis the Community District boundary was used.

\emph{Source Format \& Ingestion Method}

The data source was downloaded in a zipfile containing multiple csv
files for various geographic boundaries, the data was extracted, and a
singular csv file containing housing units by Community District was
ingested for processing.

\textbf{- Dataset 3:
\href{https://data.cityofnewyork.us/City-Government/NYC-Parks-Districts/mebz-ditc/about_data}{NYC
Parks Disticts}}

\emph{Overview \& Data Assumptions}

This data set contained both the NYC Parks Department districts and
Community Districts within one file. This geography data was processed
so as to flatten the number of Community Districts that were listed for
each respective Parks District. For example, if one row of data for
Parks District X listed overlaps with Community Districts A, B and C.
These three districts, in the raw data, are listed in one cell. This
data was parsed so that the one row was extracted into three different
rows, having one row for each Community District associated with the
Parks District. The data was essentially flattened long ways, the data
was subsequently grouped to get a unique value count for Community
Districts per Parks District. The processed table was used as a
crosswalk to process the syringe numbers to obtain an estimate of
syringe counts at the Community District level. As mentioned, the NYC
Syringe Collection data only contained Parks districts, this crosswalk
was joined into the syringe data in order to get syringe count estimates
for Community Districts. This is discussed further in the methodology
section.

\emph{Source Format \& Ingestion Method}

Read in directly from a single CSV URL.

\textbf{- Dataset 4:
\href{https://data.cityofnewyork.us/Social-Services/Individual-Census-by-Borough-Community-District-an/veav-vj3r/about_data}{NYC
Dept. Homelessness Services Individual Census by Borough, Community
District, and Facility Type}}

\emph{Overview \& Data Assumptions}

This dataset contains counts of individuals within the various types of
shelters across New York City by community district for various
reporting dates. The shelter types covered by this dataset include:

\begin{itemize}
\item
  Adult Family (Commercial Hotel): The population of adults in makeshift
  homeless shelters held within commercial hotels designated for adult
  families (i.e.m married couples with no children, a family with no
  children under the age of 21, or an unmarried couple who meets the DHS
  definition of a family unit).
\item
  Adult Family Shelter: The population of adults in homeless shelters
  designated for adult families (i.e.m married couples with no children,
  a family with no children under the age of 21, or an unmarried couple
  who meets the DHS definition of a family unit).
\item
  Adult Shelter: The population of adults in homeless shelters
  designated for single unhoused adults.
\item
  Adult Shelter (Commercial Hotel): The population of adults in
  makeshift homeless shelters held within commercial hotels designated
  for single unhoused adults.
\item
  Family with Children (Commercial Hotel): The population of individuals
  in makeshift homeless shelters held within commercial hotels
  designated for families with children.
\item
  Family with Children Shelter: The population of individuals in
  homeless shelters designated for families with children.
\end{itemize}

For the sake of this analysis, only those shelters categories with
adult-only numbers were counted, the assumption being is that
intravenous drug users would be adults and not families with children.
This means that the totals for Adult Family (Commercial Hotel), Adult
Family Shelter, Adult Shelter, and Adult Shelter (Commercial Hotel) were
used in the analysis for homeless populations within a community
district. Those columns that had counts for family-only specific counts
were not used.

\emph{Source Format \& Ingestion Method}

Sourced from JSON formatted files. The NYC Open Data API was used for
iteratively ingesting this data from JSON structured formatting.

\subsection{Methodology}\label{methodology}

After ingesting the various data sets through their respective means,
the data was processed. Each data set needed a unique series of
processing steps in order to yield the finalized version of the data
ultimately included in this analysis.

\paragraph{Syringe Data Processing}\label{syringe-data-processing}

The Syringe data had totals for each park district on each date that the
syringes were collected. For the sake of this analysis, the numbers were
aggregated up to an annual collection total for each park district.
Using the processed Community District to Park District crosswalk data,
the final version of the crosswalk contained Community District, Parks
District, and a count of distinct Community Districts for each parks
district. The processed versions of both dfs were joined, and in order
to yield an estimate for ground syringes recovered the Park District
Syringe totals, which were native to the data, were divided by the
number of community districts overlapping with each Parks District to
generate a syringe estimate for each community district.

\paragraph{Controlling for Varying
Geographies}\label{controlling-for-varying-geographies}

As mentioned, the incongruities of the geographic units between the
Community District-based housing and homeless datasets and the NYC Parks
District syringe collection data set mandated the raw syringe totals be
processed to yield a syringe estimate for Community District. Community
Districts are a city-wide administrative boundary associated with the
city's community boards, while the parks districts are an internal NYC
Parks Department administrative boundary. These two geographic
boundaries do not have a 1:1 relationship with each other. There were
instances of a singular community district overlapping with multiple
parks districts, and vice versa. In order to approximate the number of
syringes per community district, I simply divided the total number of
syringes for each parks drastic by the total number of distinct
Community Districts in each respective parks district. While this
methodology is imperfect, the syringe collection data did not have
Latitude and Longitude for each collected syringe or for specific
collection sites, so a spatial join using shapefile geographies could
not be completed.

\paragraph{Housing Data}\label{housing-data}

The housing data that was used was imported after downloading a zip file
from 2023 Q4, and using one Community District-specific CSV from the
batch of files. The CSV used contains annual counts fr the number of
completed housing constructions projects fr each year from 2010 through
2023 by community district. For this analysis the assumption was made
that the number of syringes collected in an area, provided that syringe
collection is impacted by new housing units, would be a lagging
indicator. With this mindset, the number of completed housing
construction projects in a community district for five years prior to
the year the syringes were collected, was the aggregate version of
housing numbers used for the analysis In other words, for each year in
the syringe collection data, the preceding five years of construction
jobs were summed up and the average Year-Over-Year change in
construction jobs for those years were calculated. For instance, for the
total number of estimated syringes collected for a Community District in
2017, the metrics for housing-related construction jobs within that same
community district would be the aggregate sum of all housing jobs from
2011 through 2016, as well as the average Year-Over-Year change in
housing jobs for those years. This aggregation into two main columns for
each annually summed community district row allowed for the housing
numbers to be joined with the Syringe Collection data.

\paragraph{Homeless Population Data}\label{homeless-population-data}

The Homeless shelter population data was pulled into the analysis in
JSON format via NYC Open Data's API. To process this data the sum of the
following columns were used as an estimate for the total number of
unhoused adults: adult\_shelter, adult\_shelter\_comm\_hotel,
adult\_family\_shelter,and adult\_family\_comm\_hotel. These columns
were counts from various types of shelters throughout the city. The
row-specific sums for these columns were then averaged for each
community district on an annual basis, which was to control for multiple
collections within a year. Once completed, the processed data was joined
into the df that contained the syringes data and the housing
construction data.

\subsection{Ingestion}\label{ingestion}

\paragraph{NYC Parks Syringe Collection
Data}\label{nyc-parks-syringe-collection-data}

\begin{Shaded}
\begin{Highlighting}[]
\DocumentationTok{\#\# API Intake; Documentation states there are 35,155 rows of data. Iterate through these numbers for API intake ( Limited to 1,000 row chunks)}
\NormalTok{total\_srg\_rows }\OtherTok{\textless{}{-}} \DecValTok{35155} \CommentTok{\#Manually Checked on Site}
\NormalTok{srg\_endpnt }\OtherTok{\textless{}{-}} \StringTok{"https://data.cityofnewyork.us/resource/t8xi{-}d5wb.csv"}
\NormalTok{srg\_suffix }\OtherTok{\textless{}{-}} \StringTok{\textquotesingle{}?$offset=\textquotesingle{}}

\DocumentationTok{\#\#Initial Pull}
\NormalTok{initial\_pull }\OtherTok{\textless{}{-}} \FunctionTok{read.table}\NormalTok{(srg\_endpnt, }\AttributeTok{header =} \ConstantTok{TRUE}\NormalTok{, }\AttributeTok{sep =} \StringTok{","}\NormalTok{, }\AttributeTok{dec =} \StringTok{"."}\NormalTok{)}
\NormalTok{offset }\OtherTok{\textless{}{-}}\FunctionTok{nrow}\NormalTok{(initial\_pull) }\CommentTok{\# Should be the length of the API chunk limit. (e.g., 1000 rows)}

\DocumentationTok{\#\#Iteratting through all of the chunks \& Appending to empty DF}
\NormalTok{running\_pulls }\OtherTok{\textless{}{-}} \FunctionTok{data.frame}\NormalTok{()}
\ControlFlowTok{for}\NormalTok{ (i }\ControlFlowTok{in} \FunctionTok{seq}\NormalTok{(}\DecValTok{0}\NormalTok{, total\_srg\_rows, }\AttributeTok{by =}\NormalTok{ offset)) \{}
  \ControlFlowTok{if}\NormalTok{ (i }\SpecialCharTok{==} \DecValTok{0}\NormalTok{)\{}
    \ControlFlowTok{next}
\NormalTok{  \}}
\NormalTok{  interim\_endpoint }\OtherTok{\textless{}{-}} \FunctionTok{paste}\NormalTok{(srg\_endpnt,srg\_suffix,i,}\AttributeTok{sep =} \StringTok{""}\NormalTok{)}
  \CommentTok{\# print(interim\_endpoint)}
\NormalTok{  interim\_pull }\OtherTok{\textless{}{-}} \FunctionTok{read.table}\NormalTok{(interim\_endpoint, }\AttributeTok{header =} \ConstantTok{TRUE}\NormalTok{, }\AttributeTok{sep =} \StringTok{","}\NormalTok{, }\AttributeTok{dec =} \StringTok{"."}\NormalTok{)}
\NormalTok{  running\_pulls }\OtherTok{\textless{}{-}} \FunctionTok{rbind}\NormalTok{(running\_pulls,interim\_pull)}
\NormalTok{\}}

\DocumentationTok{\#\#Putting all Srynge Pulls together }
\NormalTok{raw\_syrg }\OtherTok{\textless{}{-}} \FunctionTok{rbind}\NormalTok{(initial\_pull,running\_pulls)}

\DocumentationTok{\#\#confirming number of total rows}
\CommentTok{\# print(nrow(initial\_pull))}
\CommentTok{\# print(nrow(running\_pulls))}
\CommentTok{\# print(nrow(raw\_syrg))}
\end{Highlighting}
\end{Shaded}

\paragraph{NYC Dept. City Planning Housing
Database}\label{nyc-dept.-city-planning-housing-database}

\begin{Shaded}
\begin{Highlighting}[]
\DocumentationTok{\#\# Dataset 2: NYC DCP Housing Data (Broken Down by Multiple Geographic Area\textquotesingle{}s; Choosing Community District Breakdown)}

\DocumentationTok{\#\#\# Downloading the 23Q4 CSV Zip File from the [DCP Wesbite](https://www.nyc.gov/site/planning/data{-}maps/open{-}data/dwn{-}housing{-}database.page\#housingdevelopmentproject).}
\DocumentationTok{\#\#\# Selected "HousingDB\_by\_CommunityDistrict.csv" from zipped contents.}
\FunctionTok{download.file}\NormalTok{(}\StringTok{"https://s{-}media.nyc.gov/agencies/dcp/assets/files/zip/data{-}tools/bytes/nychousingdb\_23q4\_csv.zip"}\NormalTok{,}
              \AttributeTok{destfile =} \StringTok{"housing\_zip.zip"}\NormalTok{, }\AttributeTok{mode =} \StringTok{"wb"}\NormalTok{)}
\FunctionTok{unzip}\NormalTok{(}\StringTok{"housing\_zip.zip"}\NormalTok{, }\AttributeTok{files =} \StringTok{"HousingDB\_by\_CommunityDistrict.csv"}\NormalTok{, }\AttributeTok{exdir =} \FunctionTok{tempdir}\NormalTok{())}
\NormalTok{housing\_path }\OtherTok{\textless{}{-}} \FunctionTok{file.path}\NormalTok{(}\FunctionTok{tempdir}\NormalTok{(), }\StringTok{"HousingDB\_by\_CommunityDistrict.csv"}\NormalTok{)}
\NormalTok{raw\_housing }\OtherTok{\textless{}{-}} \FunctionTok{read.table}\NormalTok{(housing\_path, }\AttributeTok{header =} \ConstantTok{TRUE}\NormalTok{, }\AttributeTok{sep =} \StringTok{","}\NormalTok{, }\AttributeTok{dec =} \StringTok{"."}\NormalTok{)}
\end{Highlighting}
\end{Shaded}

\paragraph{NYC Parks Disticts}\label{nyc-parks-disticts}

\begin{Shaded}
\begin{Highlighting}[]
\DocumentationTok{\#\# Dataset 3: NYC Parks DIstricts to Community District Crosswalk (https://data.cityofnewyork.us/City{-}Government/NYC{-}Parks{-}Districts/mebz{-}ditc/about\_data)}
\NormalTok{crsswlk}\OtherTok{\textless{}{-}} \FunctionTok{read.table}\NormalTok{(}\StringTok{"https://data.cityofnewyork.us/resource/mebz{-}ditc.csv"}\NormalTok{,}\AttributeTok{header =} \ConstantTok{TRUE}\NormalTok{, }\AttributeTok{sep =} \StringTok{","}\NormalTok{, }\AttributeTok{dec =} \StringTok{"."}\NormalTok{)}
\end{Highlighting}
\end{Shaded}

\paragraph{NYC Dept. Homelessness Services Individual Census by Borough,
Community District, and Facility
Type}\label{nyc-dept.-homelessness-services-individual-census-by-borough-community-district-and-facility-type}

\begin{Shaded}
\begin{Highlighting}[]
\DocumentationTok{\#\# Pulling in Via JSON API }
\CommentTok{\# Endpoint details}
\NormalTok{total\_rows }\OtherTok{\textless{}{-}} \DecValTok{4325} \CommentTok{\#Manually found row count on site}
\NormalTok{hmls\_endpnt }\OtherTok{\textless{}{-}} \StringTok{"https://data.cityofnewyork.us/resource/veav{-}vj3r.json"}
\NormalTok{hmls\_suffix }\OtherTok{\textless{}{-}} \StringTok{"?$offset="}

\CommentTok{\# Initial pull to determine chunk size}
\NormalTok{initial\_pull }\OtherTok{\textless{}{-}} \FunctionTok{fromJSON}\NormalTok{(}\FunctionTok{paste0}\NormalTok{(hmls\_endpnt, hmls\_suffix, }\DecValTok{0}\NormalTok{)) }\CommentTok{\#Fetch the first chunk}

\NormalTok{offset }\OtherTok{\textless{}{-}} \FunctionTok{nrow}\NormalTok{(initial\_pull) }\CommentTok{\# Number of rows in each chunk returned by the API}
\ControlFlowTok{if}\NormalTok{ (}\FunctionTok{is.null}\NormalTok{(offset) }\SpecialCharTok{||}\NormalTok{ offset }\SpecialCharTok{==} \DecValTok{0}\NormalTok{) \{}
  \FunctionTok{stop}\NormalTok{(}\StringTok{"Initial pull returned no rows. Check the API endpoint or offset logic."}\NormalTok{)}
\NormalTok{\}}

\CommentTok{\# List to store the chunks}
\NormalTok{running\_pulls }\OtherTok{\textless{}{-}} \FunctionTok{list}\NormalTok{()}
\CommentTok{\# Loop through offsets to fetch all chunks}
\ControlFlowTok{for}\NormalTok{ (i }\ControlFlowTok{in} \FunctionTok{seq}\NormalTok{(}\DecValTok{0}\NormalTok{, total\_rows, }\AttributeTok{by =}\NormalTok{ offset)) \{}
  \CommentTok{\# Construct endpoint with offset}
\NormalTok{  interim\_endpoint }\OtherTok{\textless{}{-}} \FunctionTok{paste0}\NormalTok{(hmls\_endpnt, hmls\_suffix, i)}
  \CommentTok{\# print(paste("Fetching data from:", interim\_endpoint))}
  
  \CommentTok{\# Fetch data from the current offset}
\NormalTok{  interim\_pull }\OtherTok{\textless{}{-}} \FunctionTok{fromJSON}\NormalTok{(interim\_endpoint)}
  \DocumentationTok{\#\# Ensuring Zeros instead of nulls}
\NormalTok{  interim\_pull[}\FunctionTok{is.na}\NormalTok{(interim\_pull)] }\OtherTok{\textless{}{-}} \DecValTok{0}
  \DocumentationTok{\#\# Keeping the columns we want (Only Adult Counts)}
\NormalTok{  interim\_pull}\OtherTok{\textless{}{-}}\NormalTok{interim\_pull[}\FunctionTok{c}\NormalTok{(}\StringTok{"report\_date"}\NormalTok{,}\StringTok{"borough"}\NormalTok{,}\StringTok{"community\_districts"}\NormalTok{,}\StringTok{"adult\_shelter"}\NormalTok{,}
                 \StringTok{"adult\_shelter\_comm\_hotel"}\NormalTok{, }\StringTok{"adult\_family\_shelter"}\NormalTok{,}\StringTok{"adult\_family\_comm\_hotel"}\NormalTok{)]}
  \CommentTok{\# Check if the pull is empty (no more rows)}
  \ControlFlowTok{if}\NormalTok{ (}\FunctionTok{length}\NormalTok{(interim\_pull) }\SpecialCharTok{==} \DecValTok{0}\NormalTok{) \{}
    \CommentTok{\# print("No more data to fetch.")}
    \ControlFlowTok{break}
\NormalTok{  \}}
  
  \CommentTok{\# Append the data to the list}
\NormalTok{  running\_pulls[[}\FunctionTok{length}\NormalTok{(running\_pulls) }\SpecialCharTok{+} \DecValTok{1}\NormalTok{]] }\OtherTok{\textless{}{-}}\NormalTok{ interim\_pull}
\NormalTok{\}}

\CommentTok{\# Combine all fetched data into a single data frame}
\NormalTok{raw\_hmls }\OtherTok{\textless{}{-}} \FunctionTok{do.call}\NormalTok{(rbind, running\_pulls)}

\CommentTok{\# Confirming the total number of rows}
 \CommentTok{\#print(paste("Total rows fetched:", nrow(raw\_hmls))) \#\# Shuld be4325}

\DocumentationTok{\#\# Filling in Borough values based on Park District}
\NormalTok{raw\_hmls\_processed}\OtherTok{\textless{}{-}}\NormalTok{ raw\_hmls }\SpecialCharTok{\%\textgreater{}\%}
  \FunctionTok{filter}\NormalTok{(borough}\SpecialCharTok{!=}\StringTok{"Westchester"}\NormalTok{)}\SpecialCharTok{\%\textgreater{}\%}
  \FunctionTok{mutate}\NormalTok{(}\AttributeTok{cd\_boro\_num =} \FunctionTok{case\_when}\NormalTok{(borough}\SpecialCharTok{==} \StringTok{"Bronx"}\SpecialCharTok{\textasciitilde{}} \StringTok{"2"}\NormalTok{,}
\NormalTok{              borough}\SpecialCharTok{==} \StringTok{"Queens"} \SpecialCharTok{\textasciitilde{}}\StringTok{"4"}\NormalTok{,}
\NormalTok{              borough}\SpecialCharTok{==} \StringTok{"Manhattan"} \SpecialCharTok{\textasciitilde{}}\StringTok{"1"}\NormalTok{,}
\NormalTok{              borough}\SpecialCharTok{==} \StringTok{"Brooklyn"}\SpecialCharTok{\textasciitilde{}}\StringTok{"3"}\NormalTok{,}
\NormalTok{              borough}\SpecialCharTok{==}\StringTok{"Staten Island"}\SpecialCharTok{\textasciitilde{}}\StringTok{"5"}\NormalTok{),}
         \AttributeTok{padded\_cd =} \FunctionTok{str\_pad}\NormalTok{(community\_districts, }\AttributeTok{width =} \DecValTok{2}\NormalTok{, }\AttributeTok{side =} \StringTok{"left"}\NormalTok{, }\AttributeTok{pad =} \StringTok{"0"}\NormalTok{),}
         \AttributeTok{data\_year =} \FunctionTok{year}\NormalTok{(}\FunctionTok{ymd\_hms}\NormalTok{(report\_date))) }\SpecialCharTok{\%\textgreater{}\%}
  \FunctionTok{mutate}\NormalTok{(}\AttributeTok{full\_community\_district =} \FunctionTok{paste0}\NormalTok{(cd\_boro\_num,padded\_cd))}


\DocumentationTok{\#\# Data Type Conversion before Sum}
\NormalTok{raw\_hmls\_processed[, }\FunctionTok{c}\NormalTok{(}\StringTok{"adult\_shelter"}\NormalTok{, }\StringTok{"adult\_shelter\_comm\_hotel"}\NormalTok{, }\StringTok{"adult\_family\_shelter"}\NormalTok{,}\StringTok{"adult\_family\_comm\_hotel"}\NormalTok{)] }\OtherTok{\textless{}{-}} \FunctionTok{lapply}\NormalTok{(raw\_hmls\_processed[, }\FunctionTok{c}\NormalTok{(}\StringTok{"adult\_shelter"}\NormalTok{, }\StringTok{"adult\_shelter\_comm\_hotel"}\NormalTok{, }\StringTok{"adult\_family\_shelter"}\NormalTok{,}\StringTok{"adult\_family\_comm\_hotel"}\NormalTok{)], }\ControlFlowTok{function}\NormalTok{(x) \{}
  \FunctionTok{as.numeric}\NormalTok{(x)\})}

\NormalTok{raw\_hmls\_processed}\SpecialCharTok{$}\NormalTok{AdultHomelessCount }\OtherTok{\textless{}{-}} \FunctionTok{rowSums}\NormalTok{(raw\_hmls\_processed[, }\FunctionTok{c}\NormalTok{(}\StringTok{"adult\_shelter"}\NormalTok{, }\StringTok{"adult\_shelter\_comm\_hotel"}\NormalTok{,}
                                                        \StringTok{"adult\_family\_shelter"}\NormalTok{,}\StringTok{"adult\_family\_comm\_hotel"}\NormalTok{)])}
\CommentTok{\# limiting to the columns needed}
\NormalTok{raw\_hmls\_limited }\OtherTok{\textless{}{-}}\NormalTok{ raw\_hmls\_processed[,}\FunctionTok{c}\NormalTok{(}\StringTok{"data\_year"}\NormalTok{,}\StringTok{"borough"}\NormalTok{,}\StringTok{"full\_community\_district"}\NormalTok{,}\StringTok{"AdultHomelessCount"}\NormalTok{)]}
\DocumentationTok{\#\# Grouping for a 1:1 value on year and CD, averaging th values where there are multple.}
\NormalTok{hmls\_final }\OtherTok{\textless{}{-}}\NormalTok{ raw\_hmls\_limited }\SpecialCharTok{\%\textgreater{}\%}
  \FunctionTok{group\_by}\NormalTok{(data\_year,borough, full\_community\_district) }\SpecialCharTok{\%\textgreater{}\%}
  \FunctionTok{summarise}\NormalTok{(}
    \AttributeTok{avg\_homeless\_count =} \FunctionTok{mean}\NormalTok{(AdultHomelessCount, }\AttributeTok{na.rm=}\ConstantTok{TRUE}\NormalTok{) )}

\NormalTok{hmls\_final}\OtherTok{\textless{}{-}}\NormalTok{hmls\_final }\SpecialCharTok{\%\textgreater{}\%}\NormalTok{ dplyr}\SpecialCharTok{::}\FunctionTok{rename}\NormalTok{(}\StringTok{"communitydistrict"}\OtherTok{=}\StringTok{"full\_community\_district"}\NormalTok{,}
                                  \StringTok{"year"}\OtherTok{=}\StringTok{"data\_year"}\NormalTok{)}
\end{Highlighting}
\end{Shaded}

\subsection{Processing}\label{processing}

\paragraph{Starting with Syringe Data and Geographic
Crosswalk}\label{starting-with-syringe-data-and-geographic-crosswalk}

\emph{Flattening the crosswalk longways. Multiple Community District
Values in for a singular Parks District Row. Once flattened, the table
can be used to for Park Districts to Community Districts to enrich the
Syringe Data with Community District.}

\begin{Shaded}
\begin{Highlighting}[]
\DocumentationTok{\#\# Firstly, limiting to the columns i need.}
\NormalTok{crsswlk\_lim }\OtherTok{\textless{}{-}}\NormalTok{ crsswlk[,}\FunctionTok{c}\NormalTok{(}\StringTok{"borough"}\NormalTok{,}\StringTok{"communityboard"}\NormalTok{,}\StringTok{"parkdistrict"}\NormalTok{)] }

\DocumentationTok{\#\# The NYC Parks DIstricts are not the same (1:1) as the Community Districts. Multiple community Districts for each park district. Need to flatten.}
\NormalTok{crsswlk\_parsed }\OtherTok{\textless{}{-}} \FunctionTok{data.frame}\NormalTok{()}
\ControlFlowTok{for}\NormalTok{ (i }\ControlFlowTok{in} \DecValTok{1}\SpecialCharTok{:}\FunctionTok{nrow}\NormalTok{(crsswlk\_lim)) \{}
\NormalTok{  row }\OtherTok{\textless{}{-}}\NormalTok{ crsswlk\_lim[i, ]}
\NormalTok{  cd\_raw\_char }\OtherTok{\textless{}{-}} \FunctionTok{as.character}\NormalTok{(row}\SpecialCharTok{$}\NormalTok{communityboard)}
\NormalTok{  split\_values }\OtherTok{\textless{}{-}} \FunctionTok{substring}\NormalTok{(cd\_raw\_char, }\FunctionTok{seq}\NormalTok{(}\DecValTok{1}\NormalTok{, }\FunctionTok{nchar}\NormalTok{(cd\_raw\_char), }\DecValTok{3}\NormalTok{), }\FunctionTok{seq}\NormalTok{(}\DecValTok{3}\NormalTok{, }\FunctionTok{nchar}\NormalTok{(cd\_raw\_char), }\DecValTok{3}\NormalTok{))}

\NormalTok{  expanded\_df }\OtherTok{\textless{}{-}} \FunctionTok{data.frame}\NormalTok{(}
    \AttributeTok{communitydistrict =}\NormalTok{ split\_values,}
    \AttributeTok{parkdistrict =}\NormalTok{ row}\SpecialCharTok{$}\NormalTok{parkdistrict,}
    \AttributeTok{brough =}\NormalTok{ row}\SpecialCharTok{$}\NormalTok{borough}
\NormalTok{  )}
\NormalTok{  crsswlk\_parsed }\OtherTok{\textless{}{-}} \FunctionTok{rbind}\NormalTok{(crsswlk\_parsed, expanded\_df)\}}

\CommentTok{\# Sorting DF}
\NormalTok{crsswlk\_parsed }\OtherTok{\textless{}{-}}\NormalTok{ crsswlk\_parsed }\SpecialCharTok{\%\textgreater{}\%} \FunctionTok{arrange}\NormalTok{(communitydistrict)}
\end{Highlighting}
\end{Shaded}

\emph{Grouping Data to get Total Counts of Community District per Parks
District}

\begin{Shaded}
\begin{Highlighting}[]
\DocumentationTok{\#\# Getting Denominator for Aggregate Park District Syringe Totals; The number of Community Districts within each Park District. }
\NormalTok{crsswlk\_div\_num\_cd }\OtherTok{\textless{}{-}}\NormalTok{ crsswlk\_parsed }\SpecialCharTok{\%\textgreater{}\%} 
                    \FunctionTok{group\_by}\NormalTok{(parkdistrict) }\SpecialCharTok{\%\textgreater{}\%}
                    \FunctionTok{summarise}\NormalTok{(}\AttributeTok{cd\_count\_for\_pd =} \FunctionTok{n\_distinct}\NormalTok{(communitydistrict))}

\DocumentationTok{\#\# Crosswalk Final; CD Counts added, and data grouped. }
\NormalTok{crosswalk\_final }\OtherTok{\textless{}{-}} \FunctionTok{merge}\NormalTok{(crsswlk\_parsed,crsswlk\_div\_num\_cd, }\AttributeTok{by =} \StringTok{"parkdistrict"}\NormalTok{, }\AttributeTok{all =} \ConstantTok{FALSE}\NormalTok{)}
\end{Highlighting}
\end{Shaded}

\emph{Further Cleaning, Null Removal, and Limiting to what is Needed}

\begin{Shaded}
\begin{Highlighting}[]
\DocumentationTok{\#\# Removing those entries that have no value for a park district }
\CommentTok{\# print(nrow(raw\_syrg)) \#35,155}
\NormalTok{syringe\_lim }\OtherTok{\textless{}{-}}\NormalTok{ raw\_syrg }\SpecialCharTok{\%\textgreater{}\%} \FunctionTok{filter}\NormalTok{(district }\SpecialCharTok{!=} \StringTok{""}\NormalTok{) }

\DocumentationTok{\#\# Filling in Borough values based on Park District}
\NormalTok{syringe\_lim}\OtherTok{\textless{}{-}}\NormalTok{ syringe\_lim }\SpecialCharTok{\%\textgreater{}\%}
 \FunctionTok{mutate}\NormalTok{(}\AttributeTok{borough =} \FunctionTok{ifelse}\NormalTok{(}
\NormalTok{    borough }\SpecialCharTok{==} \StringTok{""}\NormalTok{,  }\CommentTok{\# Check if \textasciigrave{}category\textasciigrave{} is blank}
    \FunctionTok{case\_when}\NormalTok{(}
      \FunctionTok{substr}\NormalTok{(district, }\DecValTok{1}\NormalTok{, }\DecValTok{1}\NormalTok{) }\SpecialCharTok{==} \StringTok{"X"} \SpecialCharTok{\textasciitilde{}} \StringTok{"Bronx"}\NormalTok{,}
      \FunctionTok{substr}\NormalTok{(district, }\DecValTok{1}\NormalTok{, }\DecValTok{1}\NormalTok{) }\SpecialCharTok{==} \StringTok{"M"} \SpecialCharTok{\textasciitilde{}} \StringTok{"Manhattan"}\NormalTok{,}
      \FunctionTok{substr}\NormalTok{(district, }\DecValTok{1}\NormalTok{, }\DecValTok{1}\NormalTok{) }\SpecialCharTok{==} \StringTok{"Q"} \SpecialCharTok{\textasciitilde{}} \StringTok{"Queens"}\NormalTok{,}
      \FunctionTok{substr}\NormalTok{(district, }\DecValTok{1}\NormalTok{, }\DecValTok{1}\NormalTok{) }\SpecialCharTok{==} \StringTok{"B"} \SpecialCharTok{\textasciitilde{}} \StringTok{"Brooklyn"}\NormalTok{,}
      \FunctionTok{substr}\NormalTok{(district, }\DecValTok{1}\NormalTok{, }\DecValTok{1}\NormalTok{) }\SpecialCharTok{==} \StringTok{"R"} \SpecialCharTok{\textasciitilde{}} \StringTok{"Staten Island"}\NormalTok{,}
\NormalTok{    ),}
\NormalTok{    borough}
\NormalTok{  ))}

\CommentTok{\#print(nrow(syringe\_lim)) \#35,060 (95 Rows Removed)}

\DocumentationTok{\#\#\# Limiting the raw syringe data to the columns i need for my analysis}
\NormalTok{syringe\_lim }\OtherTok{\textless{}{-}}\NormalTok{ syringe\_lim[,}\FunctionTok{c}\NormalTok{(}\StringTok{"year"}\NormalTok{,}\StringTok{"group"}\NormalTok{,}\StringTok{"location"}\NormalTok{,}\StringTok{"borough"}\NormalTok{,}\StringTok{"district"}\NormalTok{,}\StringTok{"property\_type"}\NormalTok{,}\StringTok{"ground\_syringes"}\NormalTok{,}\StringTok{"kiosk\_syringes"}\NormalTok{,}\StringTok{"total\_syringes"}\NormalTok{)]}
\end{Highlighting}
\end{Shaded}

\emph{Syringe Approximation by Community District}

\begin{Shaded}
\begin{Highlighting}[]
\NormalTok{syringe\_grouped }\OtherTok{\textless{}{-}}\NormalTok{ syringe\_lim }\SpecialCharTok{\%\textgreater{}\%} 
  \FunctionTok{group\_by}\NormalTok{(year,borough, district) }\SpecialCharTok{\%\textgreater{}\%}
  \FunctionTok{summarise}\NormalTok{(}
    \AttributeTok{total\_syringes =} \FunctionTok{sum}\NormalTok{(total\_syringes, }\AttributeTok{na.rm=}\ConstantTok{TRUE}\NormalTok{),}
    \AttributeTok{ground\_syringes=} \FunctionTok{sum}\NormalTok{(ground\_syringes, }\AttributeTok{na.rm=}\ConstantTok{TRUE}\NormalTok{)}
\NormalTok{  )}
\CommentTok{\# print(head(syringe\_grouped))}

\CommentTok{\#renaming columns as needed }
\NormalTok{syringe\_grouped }\OtherTok{\textless{}{-}}\NormalTok{ syringe\_grouped }\SpecialCharTok{\%\textgreater{}\%}\NormalTok{ dplyr}\SpecialCharTok{::}\FunctionTok{rename}\NormalTok{(}\AttributeTok{parkdistrict =}\NormalTok{ district)}

\DocumentationTok{\#\# Looking at min and max years}
\CommentTok{\# print(min(syringe\_grouped$year))\#2017}
\CommentTok{\# print(max(syringe\_grouped$year))\#2024}

\NormalTok{syringe\_grouped\_enr }\OtherTok{\textless{}{-}} \FunctionTok{merge}\NormalTok{(syringe\_grouped, crosswalk\_final, }\AttributeTok{by =} \StringTok{"parkdistrict"}\NormalTok{, }\AttributeTok{all =} \ConstantTok{FALSE}\NormalTok{)}
\NormalTok{syringe\_massaged }\OtherTok{\textless{}{-}}\NormalTok{ syringe\_grouped\_enr }\SpecialCharTok{\%\textgreater{}\%} \FunctionTok{mutate}\NormalTok{(}\AttributeTok{ttl\_syring\_est\_interim =}\NormalTok{  total\_syringes }\SpecialCharTok{/}\NormalTok{ cd\_count\_for\_pd)}

\NormalTok{syringe\_final }\OtherTok{\textless{}{-}}\NormalTok{ syringe\_massaged }\SpecialCharTok{\%\textgreater{}\%} 
  \FunctionTok{group\_by}\NormalTok{(year,borough,communitydistrict) }\SpecialCharTok{\%\textgreater{}\%}
  \FunctionTok{summarise}\NormalTok{(}\AttributeTok{total\_syringe\_ests =} \FunctionTok{sum}\NormalTok{(ttl\_syring\_est\_interim, }\AttributeTok{na.rm=}\ConstantTok{TRUE}\NormalTok{) )}
\end{Highlighting}
\end{Shaded}

\paragraph{Starting with Housing Data}\label{starting-with-housing-data}

\emph{Summarizing data for total new Units by CD from the DOB for
various rolling 5 years periods for each of the years within the syringe
data}

\begin{Shaded}
\begin{Highlighting}[]
\DocumentationTok{\#\# Limiting Columns for is needed for analysis}
\NormalTok{raw\_housing}\OtherTok{\textless{}{-}}\NormalTok{raw\_housing[,}\FunctionTok{c}\NormalTok{(}\StringTok{"commntydst"}\NormalTok{,}\StringTok{"comp2010"}\NormalTok{,}\StringTok{"comp2011"}\NormalTok{,}\StringTok{"comp2012"}\NormalTok{,}\StringTok{"comp2013"}\NormalTok{,}\StringTok{"comp2014"}\NormalTok{,}\StringTok{"comp2015"}\NormalTok{,}\StringTok{"comp2016"}\NormalTok{,}\StringTok{"comp2017"}\NormalTok{,}
               \StringTok{"comp2018"}\NormalTok{,}\StringTok{"comp2019"}\NormalTok{,}\StringTok{"comp2020"}\NormalTok{,}\StringTok{"comp2021"}\NormalTok{,}\StringTok{"comp2022"}\NormalTok{,}\StringTok{"comp2023"}\NormalTok{)]}

\DocumentationTok{\#\# Summarizing data for totals for various rolling 5 years periods for each of the years within the syringe data}
\DocumentationTok{\#\#\#Syringe Rolling Years}
\NormalTok{housing\_agg\_df }\OtherTok{\textless{}{-}} \FunctionTok{data.frame}\NormalTok{()}
\ControlFlowTok{for}\NormalTok{ (i }\ControlFlowTok{in} \DecValTok{1}\SpecialCharTok{:}\FunctionTok{nrow}\NormalTok{(syringe\_grouped)) \{}
\NormalTok{  row }\OtherTok{\textless{}{-}}\NormalTok{ syringe\_grouped[i, ]}
  \CommentTok{\# row$year}
\NormalTok{  year\_range }\OtherTok{\textless{}{-}} \FunctionTok{as.character}\NormalTok{(}\FunctionTok{as.integer}\NormalTok{(}\FunctionTok{seq}\NormalTok{(}\AttributeTok{from =}\NormalTok{ (row}\SpecialCharTok{$}\NormalTok{year}\DecValTok{{-}5}\NormalTok{), }\AttributeTok{to =}\NormalTok{ (row}\SpecialCharTok{$}\NormalTok{year}\DecValTok{{-}1}\NormalTok{), }\AttributeTok{length.out =} \DecValTok{5}\NormalTok{)))}
  \CommentTok{\# print(year\_range)}
\NormalTok{  temp\_housing }\OtherTok{\textless{}{-}}\NormalTok{ raw\_housing[,}\FunctionTok{c}\NormalTok{(}
                                 \StringTok{"commntydst"}\NormalTok{,}
                                 \FunctionTok{glue}\NormalTok{(}\StringTok{"comp\{year\_range[1]\}"}\NormalTok{),}
                                 \FunctionTok{glue}\NormalTok{(}\StringTok{"comp\{year\_range[2]\}"}\NormalTok{),}
                                 \FunctionTok{glue}\NormalTok{(}\StringTok{"comp\{year\_range[3]\}"}\NormalTok{),}
                                 \FunctionTok{glue}\NormalTok{(}\StringTok{"comp\{year\_range[4]\}"}\NormalTok{),}
                                 \FunctionTok{glue}\NormalTok{(}\StringTok{"comp\{year\_range[5]\}"}\NormalTok{))]}
\NormalTok{  temp\_housing}\SpecialCharTok{$}\NormalTok{syringe\_yr }\OtherTok{\textless{}{-}}\NormalTok{row}\SpecialCharTok{$}\NormalTok{year}
\NormalTok{  temp\_housing }\OtherTok{\textless{}{-}}\NormalTok{ temp\_housing }\SpecialCharTok{\%\textgreater{}\%}\NormalTok{ dplyr}\SpecialCharTok{::}\FunctionTok{rename}\NormalTok{(}\StringTok{"comp\_yr1"} \OtherTok{=} \FunctionTok{glue}\NormalTok{(}\StringTok{"comp\{year\_range[1]\}"}\NormalTok{),}
                                          \StringTok{"comp\_yr2"} \OtherTok{=} \FunctionTok{glue}\NormalTok{(}\StringTok{"comp\{year\_range[2]\}"}\NormalTok{),}
                                          \StringTok{"comp\_yr3"} \OtherTok{=} \FunctionTok{glue}\NormalTok{(}\StringTok{"comp\{year\_range[3]\}"}\NormalTok{),}
                                          \StringTok{"comp\_yr4"} \OtherTok{=} \FunctionTok{glue}\NormalTok{(}\StringTok{"comp\{year\_range[4]\}"}\NormalTok{),}
                                          \StringTok{"comp\_yr5"} \OtherTok{=} \FunctionTok{glue}\NormalTok{(}\StringTok{"comp\{year\_range[5]\}"}\NormalTok{))}
\NormalTok{  housing\_agg\_df }\OtherTok{\textless{}{-}} \FunctionTok{rbind}\NormalTok{(housing\_agg\_df, temp\_housing)}
\NormalTok{\}}

\CommentTok{\#Looking at result}
\CommentTok{\#print(head(housing\_agg\_df))}

\CommentTok{\# Relative Yearly Sums for individual years within 5 yr window}
\NormalTok{housing\_sums }\OtherTok{\textless{}{-}}\NormalTok{ housing\_agg\_df }\SpecialCharTok{\%\textgreater{}\%}
  \FunctionTok{mutate}\NormalTok{(}\AttributeTok{housingsum =}\NormalTok{ comp\_yr1 }\SpecialCharTok{+}\NormalTok{ comp\_yr2 }\SpecialCharTok{+}\NormalTok{ comp\_yr3 }\SpecialCharTok{+}\NormalTok{ comp\_yr4 }\SpecialCharTok{+}\NormalTok{ comp\_yr5,}
    \AttributeTok{yoy\_yr2 =} \FunctionTok{ifelse}\NormalTok{(comp\_yr1 }\SpecialCharTok{==} \DecValTok{0}\NormalTok{, }\ConstantTok{NA}\NormalTok{, (comp\_yr2 }\SpecialCharTok{{-}}\NormalTok{ comp\_yr1) }\SpecialCharTok{/}\NormalTok{ comp\_yr1 }\SpecialCharTok{*} \DecValTok{100}\NormalTok{),}
    \AttributeTok{yoy\_yr3 =} \FunctionTok{ifelse}\NormalTok{(comp\_yr2 }\SpecialCharTok{==} \DecValTok{0}\NormalTok{, }\ConstantTok{NA}\NormalTok{, (comp\_yr3 }\SpecialCharTok{{-}}\NormalTok{ comp\_yr2) }\SpecialCharTok{/}\NormalTok{ comp\_yr2 }\SpecialCharTok{*} \DecValTok{100}\NormalTok{),}
    \AttributeTok{yoy\_yr4 =} \FunctionTok{ifelse}\NormalTok{(comp\_yr3 }\SpecialCharTok{==} \DecValTok{0}\NormalTok{, }\ConstantTok{NA}\NormalTok{, (comp\_yr4 }\SpecialCharTok{{-}}\NormalTok{ comp\_yr3) }\SpecialCharTok{/}\NormalTok{ comp\_yr3 }\SpecialCharTok{*} \DecValTok{100}\NormalTok{),}
    \AttributeTok{yoy\_yr5 =} \FunctionTok{ifelse}\NormalTok{(comp\_yr4 }\SpecialCharTok{==} \DecValTok{0}\NormalTok{, }\ConstantTok{NA}\NormalTok{, (comp\_yr5 }\SpecialCharTok{{-}}\NormalTok{ comp\_yr4) }\SpecialCharTok{/}\NormalTok{ comp\_yr4 }\SpecialCharTok{*} \DecValTok{100}\NormalTok{)) }\SpecialCharTok{\%\textgreater{}\%}
  \FunctionTok{mutate}\NormalTok{(}
    \AttributeTok{avg\_yoy\_change =} \FunctionTok{rowMeans}\NormalTok{(}\FunctionTok{select}\NormalTok{(., yoy\_yr2, yoy\_yr3, yoy\_yr4, yoy\_yr5), }\AttributeTok{na.rm =} \ConstantTok{TRUE}\NormalTok{)}
\NormalTok{  )}

\NormalTok{housing\_final }\OtherTok{\textless{}{-}}\NormalTok{housing\_sums[,}\FunctionTok{c}\NormalTok{(}\StringTok{"syringe\_yr"}\NormalTok{,}\StringTok{"commntydst"}\NormalTok{,}\StringTok{"housingsum"}\NormalTok{,}\StringTok{"avg\_yoy\_change"}\NormalTok{)] }\SpecialCharTok{\%\textgreater{}\%}
\NormalTok{                dplyr}\SpecialCharTok{::}\FunctionTok{rename}\NormalTok{(}\StringTok{"communitydistrict"}\OtherTok{=}\StringTok{"commntydst"}\NormalTok{)}
\end{Highlighting}
\end{Shaded}

\paragraph{Putting it all Together; the Final
DF}\label{putting-it-all-together-the-final-df}

\begin{Shaded}
\begin{Highlighting}[]
\DocumentationTok{\#\# Joining in the homeless data to the syringe data (full join for maps. Will limit for regressions later.)}
\NormalTok{syringe\_homeless\_fnl}\OtherTok{\textless{}{-}}\FunctionTok{merge}\NormalTok{(syringe\_final,hmls\_final, }\AttributeTok{by =} \FunctionTok{c}\NormalTok{(}\StringTok{\textquotesingle{}year\textquotesingle{}}\NormalTok{,}\StringTok{"borough"}\NormalTok{,}\StringTok{"communitydistrict"}\NormalTok{), }\AttributeTok{all =} \ConstantTok{TRUE}\NormalTok{)}

\NormalTok{housing\_final\_join }\OtherTok{\textless{}{-}}\NormalTok{housing\_final }\SpecialCharTok{\%\textgreater{}\%}\NormalTok{ dplyr}\SpecialCharTok{::}\FunctionTok{rename}\NormalTok{(}\StringTok{"year"}\OtherTok{=}\StringTok{"syringe\_yr"}\NormalTok{)}

\CommentTok{\# checking data types }
\NormalTok{syringe\_homeless\_fnl}\SpecialCharTok{$}\NormalTok{year }\OtherTok{\textless{}{-}} \FunctionTok{as.integer}\NormalTok{(syringe\_homeless\_fnl}\SpecialCharTok{$}\NormalTok{year)}
\NormalTok{syringe\_homeless\_fnl}\SpecialCharTok{$}\NormalTok{communitydistrict }\OtherTok{\textless{}{-}} \FunctionTok{as.integer}\NormalTok{(syringe\_homeless\_fnl}\SpecialCharTok{$}\NormalTok{communitydistrict)}
\NormalTok{housing\_final\_join}\SpecialCharTok{$}\NormalTok{year }\OtherTok{\textless{}{-}} \FunctionTok{as.integer}\NormalTok{(housing\_final\_join}\SpecialCharTok{$}\NormalTok{year)}
\NormalTok{housing\_final\_join}\SpecialCharTok{$}\NormalTok{communitydistrict }\OtherTok{\textless{}{-}} \FunctionTok{as.integer}\NormalTok{(housing\_final\_join}\SpecialCharTok{$}\NormalTok{communitydistrict)}

\NormalTok{housing\_syringe\_df }\OtherTok{\textless{}{-}} \FunctionTok{merge}\NormalTok{(syringe\_homeless\_fnl, }
\NormalTok{                           housing\_final\_join, }
                           \AttributeTok{by=} \FunctionTok{c}\NormalTok{(}\StringTok{"year"}\NormalTok{,}\StringTok{"communitydistrict"}\NormalTok{),}\AttributeTok{all=}\ConstantTok{TRUE}\NormalTok{)}

\NormalTok{housing\_syringe\_df}\OtherTok{\textless{}{-}}\NormalTok{housing\_syringe\_df}\SpecialCharTok{\%\textgreater{}\%}\FunctionTok{distinct}\NormalTok{()}

\NormalTok{data\_final }\OtherTok{\textless{}{-}}\NormalTok{ housing\_syringe\_df[,}\FunctionTok{c}\NormalTok{(}\StringTok{"year"}\NormalTok{,}\StringTok{"borough"}\NormalTok{,}\StringTok{"communitydistrict"}\NormalTok{,}
                                    \StringTok{"total\_syringe\_ests"}\NormalTok{,}\StringTok{"avg\_homeless\_count"}\NormalTok{,}
                                    \StringTok{"housingsum"}\NormalTok{,}\StringTok{"avg\_yoy\_change"}\NormalTok{)]}\SpecialCharTok{\%\textgreater{}\%}
\NormalTok{  dplyr}\SpecialCharTok{::}\FunctionTok{rename}\NormalTok{(}\StringTok{"prev5\_avg\_yoy\_change"}\OtherTok{=}\StringTok{"avg\_yoy\_change"}\NormalTok{)}
\end{Highlighting}
\end{Shaded}

\subsection{Analysis}\label{analysis}

Firstly, let's take a look at the data itself and see what the processed
data shows us about these three dimensions of data.

\subsubsection{Basic Summary Statistics}\label{basic-summary-statistics}

\textbf{SUmmary for Syringes}

\begin{Shaded}
\begin{Highlighting}[]
\FunctionTok{print}\NormalTok{(}\FunctionTok{summary}\NormalTok{(data\_final}\SpecialCharTok{$}\NormalTok{total\_syringe\_ests))}
\end{Highlighting}
\end{Shaded}

\begin{verbatim}
##     Min.  1st Qu.   Median     Mean  3rd Qu.     Max.     NA's 
##      0.5      7.0     55.0   4875.9   1540.8 101547.0      374
\end{verbatim}

\textbf{Summary for Avg. Homeless}

\begin{Shaded}
\begin{Highlighting}[]
\FunctionTok{print}\NormalTok{(}\FunctionTok{summary}\NormalTok{(data\_final}\SpecialCharTok{$}\NormalTok{avg\_homeless\_count))}
\end{Highlighting}
\end{Shaded}

\begin{verbatim}
##    Min. 1st Qu.  Median    Mean 3rd Qu.    Max.    NA's 
##    0.00   35.33  290.92  364.35  583.70 1681.44     157
\end{verbatim}

\textbf{Summary for Housing Sum}

\begin{Shaded}
\begin{Highlighting}[]
\FunctionTok{print}\NormalTok{(}\FunctionTok{summary}\NormalTok{(data\_final}\SpecialCharTok{$}\NormalTok{housingsum))}
\end{Highlighting}
\end{Shaded}

\begin{verbatim}
##    Min. 1st Qu.  Median    Mean 3rd Qu.    Max. 
##  -194.0   273.0   877.5  1624.7  2174.5 10451.0
\end{verbatim}

\textbf{Summary for Housing YOY Change}

\begin{Shaded}
\begin{Highlighting}[]
\FunctionTok{print}\NormalTok{(}\FunctionTok{summary}\NormalTok{(data\_final}\SpecialCharTok{$}\NormalTok{prev5\_avg\_yoy\_change))}
\end{Highlighting}
\end{Shaded}

\begin{verbatim}
##      Min.   1st Qu.    Median      Mean   3rd Qu.      Max.      NA's 
## -2973.248     4.368    38.269   232.867   110.634 22137.460        97
\end{verbatim}

\subsubsection{Histograms for
Distributions}\label{histograms-for-distributions}

\textbf{Syringe Data}

\begin{Shaded}
\begin{Highlighting}[]
\FunctionTok{ggplot}\NormalTok{(data\_final, }\FunctionTok{aes}\NormalTok{(}\AttributeTok{x =}\NormalTok{ total\_syringe\_ests)) }\SpecialCharTok{+} \FunctionTok{geom\_histogram}\NormalTok{(}\AttributeTok{binwidth =} \DecValTok{2000}\NormalTok{)}\SpecialCharTok{+} \FunctionTok{xlab}\NormalTok{(}\StringTok{"Est. Total Syringe Collected"}\NormalTok{)}
\end{Highlighting}
\end{Shaded}

\pandocbounded{\includegraphics[keepaspectratio]{FinalProject_JohnFerrara_files/figure-latex/hist_dist1a-1.pdf}}
Has a floor at Zero, cant have negative Syringes. Log Transform Needed.

\textbf{Logged Syringe Data}

\begin{Shaded}
\begin{Highlighting}[]
\FunctionTok{ggplot}\NormalTok{(data\_final, }\FunctionTok{aes}\NormalTok{(}\AttributeTok{x =} \FunctionTok{log}\NormalTok{(total\_syringe\_ests}\SpecialCharTok{+}\DecValTok{1}\NormalTok{))) }\SpecialCharTok{+} \FunctionTok{geom\_histogram}\NormalTok{(}\AttributeTok{binwidth =} \DecValTok{1}\NormalTok{) }\SpecialCharTok{+} \FunctionTok{xlab}\NormalTok{(}\StringTok{"Logged Est. Total Syringe Collected"}\NormalTok{)}
\end{Highlighting}
\end{Shaded}

\pandocbounded{\includegraphics[keepaspectratio]{FinalProject_JohnFerrara_files/figure-latex/hist_dist1b-1.pdf}}
Looks better than original.

\textbf{Adult Homeless Population}

\begin{Shaded}
\begin{Highlighting}[]
\FunctionTok{ggplot}\NormalTok{(data\_final, }\FunctionTok{aes}\NormalTok{(}\AttributeTok{x =}\NormalTok{ avg\_homeless\_count)) }\SpecialCharTok{+} \FunctionTok{geom\_histogram}\NormalTok{(}\AttributeTok{binwidth =} \DecValTok{200}\NormalTok{)}\SpecialCharTok{+} \FunctionTok{xlab}\NormalTok{(}\StringTok{"Avg. Annual Adult Homeless Population"}\NormalTok{)}
\end{Highlighting}
\end{Shaded}

\pandocbounded{\includegraphics[keepaspectratio]{FinalProject_JohnFerrara_files/figure-latex/hist_dist2a-1.pdf}}
Has a floor at zero b/c cant have negative people. Log Transform needed.

\textbf{Logged Adult Homeless Population}

\begin{Shaded}
\begin{Highlighting}[]
\FunctionTok{ggplot}\NormalTok{(data\_final, }\FunctionTok{aes}\NormalTok{(}\AttributeTok{x =} \FunctionTok{log}\NormalTok{(avg\_homeless\_count}\SpecialCharTok{+}\DecValTok{1}\NormalTok{))) }\SpecialCharTok{+} \FunctionTok{geom\_histogram}\NormalTok{(}\AttributeTok{binwidth =} \DecValTok{1}\NormalTok{)}\SpecialCharTok{+}\FunctionTok{xlab}\NormalTok{(}\StringTok{"Logged Avg. Annual Adult Homeless Population"}\NormalTok{)}
\end{Highlighting}
\end{Shaded}

\pandocbounded{\includegraphics[keepaspectratio]{FinalProject_JohnFerrara_files/figure-latex/hist_dist2b-1.pdf}}
Looks better.

\textbf{Total 5 Yr Completed Construction Projects}

\begin{Shaded}
\begin{Highlighting}[]
\FunctionTok{ggplot}\NormalTok{(data\_final, }\FunctionTok{aes}\NormalTok{(}\AttributeTok{x =}\NormalTok{ housingsum)) }\SpecialCharTok{+} \FunctionTok{geom\_histogram}\NormalTok{(}\AttributeTok{binwidth =} \DecValTok{200}\NormalTok{) }\SpecialCharTok{+} \FunctionTok{xlab}\NormalTok{(}\StringTok{"Total 5 Yr Completed Construction Projects"}\NormalTok{)}
\end{Highlighting}
\end{Shaded}

\pandocbounded{\includegraphics[keepaspectratio]{FinalProject_JohnFerrara_files/figure-latex/hist_dist3a-1.pdf}}
Has clustering at zero, trying Log Transform.

\textbf{Logged Total 5 Yr Completed Construction Projects}

\begin{Shaded}
\begin{Highlighting}[]
\FunctionTok{ggplot}\NormalTok{(data\_final, }\FunctionTok{aes}\NormalTok{(}\AttributeTok{x =} \FunctionTok{log}\NormalTok{(housingsum}\SpecialCharTok{+}\DecValTok{1}\NormalTok{))) }\SpecialCharTok{+} \FunctionTok{geom\_histogram}\NormalTok{(}\AttributeTok{binwidth =} \DecValTok{1}\NormalTok{) }\SpecialCharTok{+} \FunctionTok{xlab}\NormalTok{(}\StringTok{"Logged Total 5 Yr Completed Construction Projects"}\NormalTok{)}
\end{Highlighting}
\end{Shaded}

\pandocbounded{\includegraphics[keepaspectratio]{FinalProject_JohnFerrara_files/figure-latex/hist_dist3b-1.pdf}}
Still high count at 0, best I can do.

\textbf{Avg. 5 Yr YOY Pct. Change in Completed Construction Projects}

\begin{Shaded}
\begin{Highlighting}[]
\FunctionTok{ggplot}\NormalTok{(data\_final, }\FunctionTok{aes}\NormalTok{(}\AttributeTok{x =}\NormalTok{ prev5\_avg\_yoy\_change)) }\SpecialCharTok{+} \FunctionTok{geom\_histogram}\NormalTok{(}\AttributeTok{binwidth =} \DecValTok{500}\NormalTok{)}\SpecialCharTok{+}\FunctionTok{xlab}\NormalTok{(}\StringTok{"Avg. 5 Yr YOY Pct. Change in Completed Construction Projects"}\NormalTok{)}
\end{Highlighting}
\end{Shaded}

\pandocbounded{\includegraphics[keepaspectratio]{FinalProject_JohnFerrara_files/figure-latex/hist_dist4a-1.pdf}}

Checking the log distribution.

\textbf{Logged Avg. 5 Yr YOY Pct. Change in Completed Construction
Projects}

\begin{Shaded}
\begin{Highlighting}[]
\FunctionTok{ggplot}\NormalTok{(data\_final, }\FunctionTok{aes}\NormalTok{(}\AttributeTok{x =} \FunctionTok{log}\NormalTok{(prev5\_avg\_yoy\_change}\SpecialCharTok{+}\DecValTok{1}\NormalTok{))) }\SpecialCharTok{+} \FunctionTok{geom\_histogram}\NormalTok{(}\AttributeTok{binwidth =} \DecValTok{1}\NormalTok{) }\SpecialCharTok{+}\FunctionTok{xlab}\NormalTok{(}\StringTok{"Logged Avg. 5 Yr YOY Pct. Change in Completed Construction Projects"}\NormalTok{)}
\end{Highlighting}
\end{Shaded}

\pandocbounded{\includegraphics[keepaspectratio]{FinalProject_JohnFerrara_files/figure-latex/hist_dist4b-1.pdf}}
Log looks better for nor. dist.

After taking a look at the histograms for each of these variables that I
want to analyze, because of the zero floors for several of them and the
original distributions, Log transformations provide a more normal
distribution to analyze. I will use the log transformations in
regressions.

\subsubsection{Prepping for Map Visuals}\label{prepping-for-map-visuals}

\begin{Shaded}
\begin{Highlighting}[]
\DocumentationTok{\#\# Dealing with GeoJSON file in order to map data by Community District}
\NormalTok{cd\_geojson }\OtherTok{\textless{}{-}} \FunctionTok{tempfile}\NormalTok{(}\AttributeTok{fileext =} \StringTok{".geojson"}\NormalTok{)}
\FunctionTok{download.file}\NormalTok{(}\StringTok{"https://data.cityofnewyork.us/api/geospatial/yfnk{-}k7r4?method=export\&format=GeoJSON"}\NormalTok{,}
\NormalTok{              cd\_geojson)}
\DocumentationTok{\#\# Formatting the Community DIstrict Shapefile as needed to join with the final df.}
\NormalTok{cd\_sf }\OtherTok{\textless{}{-}} \FunctionTok{read\_sf}\NormalTok{(cd\_geojson)}
\NormalTok{cd\_sf }\OtherTok{\textless{}{-}}\NormalTok{ cd\_sf }\SpecialCharTok{\%\textgreater{}\%}\NormalTok{ dplyr}\SpecialCharTok{::}\FunctionTok{rename}\NormalTok{(}\StringTok{"communitydistrict"}\OtherTok{=}\StringTok{"boro\_cd"}\NormalTok{)}
\NormalTok{cd\_sf}\SpecialCharTok{$}\NormalTok{communitydistrict }\OtherTok{\textless{}{-}} \FunctionTok{as.integer}\NormalTok{(cd\_sf}\SpecialCharTok{$}\NormalTok{communitydistrict)}

\DocumentationTok{\#\#\# Looping through all the years in the "data\_final" df to do the join for geom, to keep the blank geoms needed for full map}
\NormalTok{adjusted\_geom }\OtherTok{=} \FunctionTok{data.frame}\NormalTok{()}
\ControlFlowTok{for}\NormalTok{ (y }\ControlFlowTok{in} \FunctionTok{unique}\NormalTok{(data\_final}\SpecialCharTok{$}\NormalTok{year))\{}
\NormalTok{  lim\_data }\OtherTok{\textless{}{-}}\NormalTok{data\_final }\SpecialCharTok{\%\textgreater{}\%} \FunctionTok{filter}\NormalTok{(year}\SpecialCharTok{==}\NormalTok{y)}
  \CommentTok{\# print(length(unique(lim\_data$communitydistrict)))}
  \CommentTok{\# Left Joining for each year, so we have Geom for all CD, even if not in the data.}
\NormalTok{  temp\_df }\OtherTok{\textless{}{-}}\NormalTok{ cd\_sf }\SpecialCharTok{\%\textgreater{}\%}
    \FunctionTok{left\_join}\NormalTok{(lim\_data, }\AttributeTok{by =} \StringTok{"communitydistrict"}\NormalTok{)}
\NormalTok{  temp\_df}\SpecialCharTok{$}\NormalTok{year }\OtherTok{\textless{}{-}}\NormalTok{ y}
\NormalTok{  adjusted\_geom}\OtherTok{\textless{}{-}}\FunctionTok{rbind}\NormalTok{(adjusted\_geom,temp\_df)}
\NormalTok{\}}

\CommentTok{\#Ensuring the borough Values arent null for those CD that are not in the Housing/Syringe Data}
\NormalTok{adjusted\_geom }\OtherTok{\textless{}{-}}\NormalTok{ adjusted\_geom }\SpecialCharTok{\%\textgreater{}\%} \FunctionTok{mutate}\NormalTok{(}\AttributeTok{borough =} \FunctionTok{ifelse}\NormalTok{(}\FunctionTok{is.na}\NormalTok{(borough),}
                                    \FunctionTok{case\_when}\NormalTok{(}
                                      \FunctionTok{substr}\NormalTok{(}\FunctionTok{as.character}\NormalTok{(communitydistrict), }\DecValTok{1}\NormalTok{, }\DecValTok{1}\NormalTok{) }\SpecialCharTok{==} \StringTok{"2"} \SpecialCharTok{\textasciitilde{}} \StringTok{"Bronx"}\NormalTok{,}
                                      \FunctionTok{substr}\NormalTok{(}\FunctionTok{as.character}\NormalTok{(communitydistrict), }\DecValTok{1}\NormalTok{, }\DecValTok{1}\NormalTok{) }\SpecialCharTok{==} \StringTok{"1"} \SpecialCharTok{\textasciitilde{}} \StringTok{"Manhattan"}\NormalTok{,}
                                      \FunctionTok{substr}\NormalTok{(}\FunctionTok{as.character}\NormalTok{(communitydistrict), }\DecValTok{1}\NormalTok{, }\DecValTok{1}\NormalTok{) }\SpecialCharTok{==} \StringTok{"4"} \SpecialCharTok{\textasciitilde{}} \StringTok{"Queens"}\NormalTok{,}
                                      \FunctionTok{substr}\NormalTok{(}\FunctionTok{as.character}\NormalTok{(communitydistrict), }\DecValTok{1}\NormalTok{, }\DecValTok{1}\NormalTok{) }\SpecialCharTok{==} \StringTok{"3"} \SpecialCharTok{\textasciitilde{}} \StringTok{"Brooklyn"}\NormalTok{,}
                                      \FunctionTok{substr}\NormalTok{(}\FunctionTok{as.character}\NormalTok{(communitydistrict), }\DecValTok{1}\NormalTok{, }\DecValTok{1}\NormalTok{) }\SpecialCharTok{==} \StringTok{"5"} \SpecialCharTok{\textasciitilde{}} \StringTok{"Staten Island"}\NormalTok{,}
\NormalTok{                                      ),}
\NormalTok{                                    borough}
\NormalTok{                                    ))}
\end{Highlighting}
\end{Shaded}

\paragraph{Mapping To Visually See The Data in a Geospatial
Context}\label{mapping-to-visually-see-the-data-in-a-geospatial-context}

\textbf{Syringe Data by Community District}

Looking at the year by year break down for each syringe collection year
in the data, one can identify that there are a decent number of nulls
within the data. This could be from either syringes not being found by
Parks staff in the vast majority of the city, or it could have to do
with when and how the data started being collected. In the earlier years
the syringe counts are limited to the South Bronx and small sections of
Manhattan, but as the years progress additional parts of the city
register a having a syringes collected.

\pandocbounded{\includegraphics[keepaspectratio]{FinalProject_JohnFerrara_files/figure-latex/mapping1-1.pdf}}

\textbf{Total Number of Housing Related Construction Projects for 5
Years Previous by Community District}

For the total number of housing related completed construction projects
registered with DOB, in the early years the high development zones are
limited to Northern Brooklyn and parts on the lower end of Manhattan. As
the years progress, the number of projects through out the city, more
specifically in Brooklyn and Queens, increase.

\pandocbounded{\includegraphics[keepaspectratio]{FinalProject_JohnFerrara_files/figure-latex/mapping2-1.pdf}}

\textbf{Average YOY Change for Housing Related Construction Projects 5
Years Previous by Community District}

The average YOY change for completed housing related construction
project is fairly consistent through out the entire city through out the
years in the data. However, the areas of what looks like Sunset Park \&
Park Slope in Brooklyn experienced noticeably high YOY change in
projects from 2017 through 2020.

\pandocbounded{\includegraphics[keepaspectratio]{FinalProject_JohnFerrara_files/figure-latex/mapping3-1.pdf}}

\textbf{Annual Average Adult Homeless Population Based On Shelter
Counts}

Lastly, for the average annual adult homeless population, there is no
data for 2017, but once data is collected for 2018 there are many
different parts of the city that have fairly high average adult homeless
populations. There doesn't seem to be any particular trend visible just
by looking at the data. However, it should be noted, that while this
data is useful for gauging the homeless population in the city, at a
Community District geographic boundary level, the means by which the
data is collected - taking counts at homeless shelters of different
kinds - implicitly causes the homeless population to be higher in areas
that have hotels or shelters designated for that population. This may
not necessarily be an issue, but it is something to keep in mind when
comparing this variable to others.

\pandocbounded{\includegraphics[keepaspectratio]{FinalProject_JohnFerrara_files/figure-latex/mapping4-1.pdf}}

\paragraph{Removing Nulls from Finalized
DF}\label{removing-nulls-from-finalized-df}

\begin{Shaded}
\begin{Highlighting}[]
\CommentTok{\#Removing the Nulls from the final data df in preparation for regression}
\NormalTok{data\_final\_nonull }\OtherTok{\textless{}{-}} \FunctionTok{na.omit}\NormalTok{(data\_final)}
\end{Highlighting}
\end{Shaded}

\subsubsection{Linear Regressions w/ Community District
Geographies}\label{linear-regressions-w-community-district-geographies}

As outlined in the introduction, there are two parts to this analysis.
The first is to see if there is a correlation between housing-related
construction projects and the homeless population. The second is to see
if the size of homeless population correlates with the amount of
syringes found in NYC parks. This first attempt at regression models
will maintain a larger number of data points constrained by community
district. However, the wide array of factors that influence various
community districts may influence the regression results obfuscating any
substantial correlation. Factors such as wealthier, gentrifying areas of
the city having more construction projects associated with their
geographic boundary, while intravenous drug use is more likely to be
concentrated in poorer, less wealthy community districts. Additionally,
housing units through out the city may not properly be shown to reduce
homelessness due to the homeless population data being sourced from
shelters. Shelters are not evenly distributed throughout the city,
making the numbers at the community district level skewed for the
overall numbers.

\subsubsection{QUESTION 1: Does the Number of Housing Related
Construction Projects for the Previous 5 Years influence the Size of
Adult Homeless
Population?}\label{question-1-does-the-number-of-housing-related-construction-projects-for-the-previous-5-years-influence-the-size-of-adult-homeless-population}

To answer this question, a linear regression analysis will look at the
relationship between the annual average for adult homeless shelter
populations and the number of completed housing construction jobs on
record with the DOB. For the housing numbers, we will look at the
average annual year-over-year (YOY) change in housing construction
projects for the five year window previous to the homeless population
year, as well as the total number of cumulative housing construction
projects completed in for the five years previous to the homeless
population year. Lastly, as outlined previously because of the original
distributions of he variable a log transform will be used in order to
get a more normal distribution for the analyzed variables.

\textbf{MODEL 1: Relationship Between Logged Adult Homeless Population
and Logged Completed Housing Construction Projects for Preceding 5 Years
at the Community District Level}

\begin{verbatim}
## 
## Call:
## lm(formula = log(avg_homeless_count + 1) ~ log(housingsum + 1), 
##     data = data_final_nonull)
## 
## Residuals:
##     Min      1Q  Median      3Q     Max 
## -5.2104 -0.6184  0.4339  1.1790  4.3159 
## 
## Coefficients:
##                     Estimate Std. Error t value Pr(>|t|)    
## (Intercept)          -4.8990     1.3206   -3.71 0.000291 ***
## log(housingsum + 1)   1.3453     0.1789    7.52 4.54e-12 ***
## ---
## Signif. codes:  0 '***' 0.001 '**' 0.01 '*' 0.05 '.' 0.1 ' ' 1
## 
## Residual standard error: 2.015 on 151 degrees of freedom
## Multiple R-squared:  0.2725, Adjusted R-squared:  0.2676 
## F-statistic: 56.55 on 1 and 151 DF,  p-value: 4.541e-12
\end{verbatim}

\pandocbounded{\includegraphics[keepaspectratio]{FinalProject_JohnFerrara_files/figure-latex/m1_plot-1.pdf}}

Model 1 \textbf{is} Statisitcally Significant, so Checking Model
Validity.

\textbf{Model 1 Linearity Check}

\begin{Shaded}
\begin{Highlighting}[]
\CommentTok{\# Linearity Check}
\FunctionTok{ggplot}\NormalTok{(m1, }\FunctionTok{aes}\NormalTok{(}\AttributeTok{x=}\NormalTok{.fitted, }\AttributeTok{y=}\NormalTok{.resid)) }\SpecialCharTok{+} 
  \FunctionTok{geom\_point}\NormalTok{()}\SpecialCharTok{+}
  \FunctionTok{geom\_hline}\NormalTok{(}\AttributeTok{yintercept =} \DecValTok{0}\NormalTok{, }\AttributeTok{linetype =} \StringTok{"dashed"}\NormalTok{) }\SpecialCharTok{+}
  \FunctionTok{labs}\NormalTok{(}\AttributeTok{x =} \StringTok{"Predicted"}\NormalTok{,}\AttributeTok{y =}\StringTok{"Residuals"}\NormalTok{)}
\end{Highlighting}
\end{Shaded}

\pandocbounded{\includegraphics[keepaspectratio]{FinalProject_JohnFerrara_files/figure-latex/m1_check1-1.pdf}}
There seems to be slight correlation in lower portion. Line seems to
have Zero Slope.

\textbf{Model 1 Residual Distribution Check 1 }

\begin{Shaded}
\begin{Highlighting}[]
\CommentTok{\# Residual Dist.}
\FunctionTok{ggplot}\NormalTok{(}\AttributeTok{data =}\NormalTok{ m1, }\FunctionTok{aes}\NormalTok{(}\AttributeTok{x =}\NormalTok{ .resid)) }\SpecialCharTok{+}\FunctionTok{geom\_histogram}\NormalTok{(}\AttributeTok{binwidth =} \DecValTok{1}\NormalTok{) }\SpecialCharTok{+} \FunctionTok{xlab}\NormalTok{(}\StringTok{"Residuals"}\NormalTok{)}
\end{Highlighting}
\end{Shaded}

\pandocbounded{\includegraphics[keepaspectratio]{FinalProject_JohnFerrara_files/figure-latex/m1_check2-1.pdf}}

\textbf{Model 1 Residual Distribution Check 2}

\begin{Shaded}
\begin{Highlighting}[]
\CommentTok{\#Variability of Constant}
\FunctionTok{ggplot}\NormalTok{(}\AttributeTok{data =}\NormalTok{ m1, }\FunctionTok{aes}\NormalTok{(}\AttributeTok{sample =}\NormalTok{ .resid)) }\SpecialCharTok{+}\FunctionTok{stat\_qq}\NormalTok{()}
\end{Highlighting}
\end{Shaded}

\pandocbounded{\includegraphics[keepaspectratio]{FinalProject_JohnFerrara_files/figure-latex/m1_check3-1.pdf}}

The residuals are basically normally distributed. Model most likely
valid.

\textbf{MODEL 2: Relationship Between Logged Adult Homeless Population
and Logged Average YOY Percent Change in Completed Housing Related
Construction Projects for Preceding 5 Years at the Community District
Level}

\begin{verbatim}
## 
## Call:
## lm(formula = log(avg_homeless_count + 1) ~ log(prev5_avg_yoy_change + 
##     1), data = data_final_nonull)
## 
## Residuals:
##     Min      1Q  Median      3Q     Max 
## -5.3665 -0.0899  0.7975  1.3047  2.0893 
## 
## Coefficients:
##                               Estimate Std. Error t value Pr(>|t|)    
## (Intercept)                    5.47574    0.75554   7.247 5.61e-11 ***
## log(prev5_avg_yoy_change + 1) -0.06501    0.18114  -0.359     0.72    
## ---
## Signif. codes:  0 '***' 0.001 '**' 0.01 '*' 0.05 '.' 0.1 ' ' 1
## 
## Residual standard error: 2.191 on 113 degrees of freedom
##   (38 observations deleted due to missingness)
## Multiple R-squared:  0.001139,   Adjusted R-squared:  -0.007701 
## F-statistic: 0.1288 on 1 and 113 DF,  p-value: 0.7203
\end{verbatim}

\pandocbounded{\includegraphics[keepaspectratio]{FinalProject_JohnFerrara_files/figure-latex/m2_plot-1.pdf}}
Model 2 is \textbf{NOT} Statistically Significant. Moving On.

\subsubsection{QUESTION 2: Does the size of the Adult Homeless
Population influence the Number of Used Syringes recovered by NYC
Parks?}\label{question-2-does-the-size-of-the-adult-homeless-population-influence-the-number-of-used-syringes-recovered-by-nyc-parks}

\textbf{MODEL 3: Relationship Between Logged Adult Homeless Population
and Logged Est. Total Number of Syringes Recovered at the Community
District Level}

\begin{verbatim}
## 
## Call:
## lm(formula = log(total_syringe_ests + 1) ~ log(avg_homeless_count + 
##     1), data = data_final_nonull)
## 
## Residuals:
##     Min      1Q  Median      3Q     Max 
## -5.4495 -2.5458 -0.1314  2.3448  6.0140 
## 
## Coefficients:
##                             Estimate Std. Error t value Pr(>|t|)    
## (Intercept)                   3.0218     0.5802   5.208 6.16e-07 ***
## log(avg_homeless_count + 1)   0.4562     0.1058   4.311 2.91e-05 ***
## ---
## Signif. codes:  0 '***' 0.001 '**' 0.01 '*' 0.05 '.' 0.1 ' ' 1
## 
## Residual standard error: 3.072 on 151 degrees of freedom
## Multiple R-squared:  0.1096, Adjusted R-squared:  0.1037 
## F-statistic: 18.59 on 1 and 151 DF,  p-value: 2.914e-05
\end{verbatim}

\pandocbounded{\includegraphics[keepaspectratio]{FinalProject_JohnFerrara_files/figure-latex/m3_plot-1.pdf}}

Model 3 is Statisitcally Significant. Checking Validity.

\textbf{Model 3 Linearity Check}

\begin{Shaded}
\begin{Highlighting}[]
\CommentTok{\# Linearity Check}
\FunctionTok{ggplot}\NormalTok{(m3, }\FunctionTok{aes}\NormalTok{(}\AttributeTok{x=}\NormalTok{.fitted, }\AttributeTok{y=}\NormalTok{.resid)) }\SpecialCharTok{+} 
  \FunctionTok{geom\_point}\NormalTok{()}\SpecialCharTok{+}
  \FunctionTok{geom\_hline}\NormalTok{(}\AttributeTok{yintercept =} \DecValTok{0}\NormalTok{, }\AttributeTok{linetype =} \StringTok{"dashed"}\NormalTok{) }\SpecialCharTok{+}
  \FunctionTok{labs}\NormalTok{(}\AttributeTok{x =} \StringTok{"Fitted values"}\NormalTok{,}\AttributeTok{y =}\StringTok{"Residuals"}\NormalTok{)}
\end{Highlighting}
\end{Shaded}

\pandocbounded{\includegraphics[keepaspectratio]{FinalProject_JohnFerrara_files/figure-latex/m3_check1-1.pdf}}
Linearity Check should be good.

\textbf{Model 3 Residual Distribution Check 1 }

\begin{Shaded}
\begin{Highlighting}[]
\CommentTok{\# Residual Dist.}
\FunctionTok{ggplot}\NormalTok{(}\AttributeTok{data =}\NormalTok{ m3, }\FunctionTok{aes}\NormalTok{(}\AttributeTok{x =}\NormalTok{ .resid)) }\SpecialCharTok{+}\FunctionTok{geom\_histogram}\NormalTok{(}\AttributeTok{binwidth =} \DecValTok{1}\NormalTok{) }\SpecialCharTok{+} \FunctionTok{xlab}\NormalTok{(}\StringTok{"Residuals"}\NormalTok{)}
\end{Highlighting}
\end{Shaded}

\pandocbounded{\includegraphics[keepaspectratio]{FinalProject_JohnFerrara_files/figure-latex/m3_check2-1.pdf}}

\textbf{Model 3 Residual Distribution Check 2}

\begin{Shaded}
\begin{Highlighting}[]
\CommentTok{\#Variablitiy of Constant}
\FunctionTok{ggplot}\NormalTok{(}\AttributeTok{data =}\NormalTok{ m3, }\FunctionTok{aes}\NormalTok{(}\AttributeTok{sample =}\NormalTok{ .resid)) }\SpecialCharTok{+}\FunctionTok{stat\_qq}\NormalTok{()}
\end{Highlighting}
\end{Shaded}

\pandocbounded{\includegraphics[keepaspectratio]{FinalProject_JohnFerrara_files/figure-latex/m3_check3-1.pdf}}

\begin{Shaded}
\begin{Highlighting}[]
\DocumentationTok{\#\# Slight Deviation b/c of right skew, but passes for normal.}
\end{Highlighting}
\end{Shaded}

Residuals are normally distributed.Model 3 most likely Valid.

\subsubsection{Linear Regressions with Aggregate Borough
Geographies}\label{linear-regressions-with-aggregate-borough-geographies}

After performing the regression analysis above, the results were
unremarkable. There were two statistically significant relationships.
Due to potential issues with the data at the Community District level,
such as an unequal distribution of parks where syringes can be found,
specific types of zoning limiting where shelters exist implicitly
skewing homeless populations, or housing projects being limited to
select areas. I decided to aggregate to the borough level instead.
Generalizing the geographic limitations may show relationships more
clearly than the more granular boundaries. The main draw back to this
aggregation is a reduction in data points, however it is worth taking a
look.

\paragraph{Aggregating the data to the borough
level.}\label{aggregating-the-data-to-the-borough-level.}

\textbf{Syringe Borough Aggregation}

\begin{Shaded}
\begin{Highlighting}[]
\NormalTok{syringe\_grouped\_boro }\OtherTok{\textless{}{-}}\NormalTok{ syringe\_lim }\SpecialCharTok{\%\textgreater{}\%} 
  \FunctionTok{group\_by}\NormalTok{(year,borough) }\SpecialCharTok{\%\textgreater{}\%}
  \FunctionTok{summarise}\NormalTok{(}
    \AttributeTok{total\_syringes =} \FunctionTok{sum}\NormalTok{(total\_syringes, }\AttributeTok{na.rm=}\ConstantTok{TRUE}\NormalTok{)}
\NormalTok{    )}
\end{Highlighting}
\end{Shaded}

\textbf{Homeless Borough Aggregation}

\begin{Shaded}
\begin{Highlighting}[]
\NormalTok{raw\_hmls\_limited\_boro }\OtherTok{\textless{}{-}}\NormalTok{ raw\_hmls\_processed[,}\FunctionTok{c}\NormalTok{(}\StringTok{"report\_date"}\NormalTok{,}\StringTok{"data\_year"}\NormalTok{,}\StringTok{"borough"}\NormalTok{,}\StringTok{"AdultHomelessCount"}\NormalTok{)]}

\DocumentationTok{\#\# Grouping by report date to sum up boro total for report date.}
\NormalTok{hmls\_grpd\_boro }\OtherTok{\textless{}{-}}\NormalTok{ raw\_hmls\_limited\_boro }\SpecialCharTok{\%\textgreater{}\%}
  \FunctionTok{group\_by}\NormalTok{(report\_date,data\_year,borough) }\SpecialCharTok{\%\textgreater{}\%}
  \FunctionTok{summarise}\NormalTok{(}
    \AttributeTok{boro\_sum\_homeless\_count =} \FunctionTok{sum}\NormalTok{(AdultHomelessCount, }\AttributeTok{na.rm=}\ConstantTok{TRUE}\NormalTok{))}

\DocumentationTok{\#\# Averaging for each boto for each year}
\NormalTok{hmls\_final\_boro }\OtherTok{\textless{}{-}}\NormalTok{ hmls\_grpd\_boro }\SpecialCharTok{\%\textgreater{}\%}
  \FunctionTok{group\_by}\NormalTok{(data\_year,borough) }\SpecialCharTok{\%\textgreater{}\%}
  \FunctionTok{summarise}\NormalTok{(}
    \AttributeTok{avg\_homeless\_count =} \FunctionTok{mean}\NormalTok{(boro\_sum\_homeless\_count, }\AttributeTok{na.rm=}\ConstantTok{TRUE}\NormalTok{))}

\NormalTok{hmls\_final\_boro }\OtherTok{\textless{}{-}}\NormalTok{ hmls\_final\_boro }\SpecialCharTok{\%\textgreater{}\%}\NormalTok{ dplyr}\SpecialCharTok{::}\FunctionTok{rename}\NormalTok{(}\AttributeTok{year=}\NormalTok{data\_year)}
\end{Highlighting}
\end{Shaded}

\textbf{Housing Borough Agg}

\begin{Shaded}
\begin{Highlighting}[]
\NormalTok{housing\_agg\_df\_boro }\OtherTok{\textless{}{-}}\NormalTok{ housing\_agg\_df }\SpecialCharTok{\%\textgreater{}\%}
  \FunctionTok{mutate}\NormalTok{(}\AttributeTok{borough =} \FunctionTok{case\_when}\NormalTok{(}
    \FunctionTok{substr}\NormalTok{(}\FunctionTok{as.character}\NormalTok{(commntydst), }\DecValTok{1}\NormalTok{, }\DecValTok{1}\NormalTok{) }\SpecialCharTok{==} \StringTok{"2"} \SpecialCharTok{\textasciitilde{}} \StringTok{"Bronx"}\NormalTok{,}
    \FunctionTok{substr}\NormalTok{(}\FunctionTok{as.character}\NormalTok{(commntydst), }\DecValTok{1}\NormalTok{, }\DecValTok{1}\NormalTok{) }\SpecialCharTok{==} \StringTok{"4"} \SpecialCharTok{\textasciitilde{}} \StringTok{"Queens"}\NormalTok{,}
    \FunctionTok{substr}\NormalTok{(}\FunctionTok{as.character}\NormalTok{(commntydst), }\DecValTok{1}\NormalTok{, }\DecValTok{1}\NormalTok{) }\SpecialCharTok{==} \StringTok{"3"} \SpecialCharTok{\textasciitilde{}} \StringTok{"Booklyn"}\NormalTok{,}
    \FunctionTok{substr}\NormalTok{(}\FunctionTok{as.character}\NormalTok{(commntydst), }\DecValTok{1}\NormalTok{, }\DecValTok{1}\NormalTok{) }\SpecialCharTok{==} \StringTok{"5"} \SpecialCharTok{\textasciitilde{}} \StringTok{"Staten Island"}\NormalTok{,}
    \FunctionTok{substr}\NormalTok{(}\FunctionTok{as.character}\NormalTok{(commntydst), }\DecValTok{1}\NormalTok{, }\DecValTok{1}\NormalTok{) }\SpecialCharTok{==} \StringTok{"1"} \SpecialCharTok{\textasciitilde{}} \StringTok{"Manhattan"}\NormalTok{))}

\NormalTok{housing\_agg\_df\_boro\_grpd}\OtherTok{\textless{}{-}}\NormalTok{ housing\_agg\_df\_boro }\SpecialCharTok{\%\textgreater{}\%}
  \FunctionTok{group\_by}\NormalTok{(syringe\_yr, borough) }\SpecialCharTok{\%\textgreater{}\%}
  \FunctionTok{summarize}\NormalTok{(}
    \AttributeTok{comp\_yr1 =} \FunctionTok{sum}\NormalTok{(comp\_yr1, }\AttributeTok{na.rm=}\ConstantTok{TRUE}\NormalTok{),}
    \AttributeTok{comp\_yr2 =} \FunctionTok{sum}\NormalTok{(comp\_yr2, }\AttributeTok{na.rm=}\ConstantTok{TRUE}\NormalTok{),}
    \AttributeTok{comp\_yr3 =} \FunctionTok{sum}\NormalTok{(comp\_yr3, }\AttributeTok{na.rm=}\ConstantTok{TRUE}\NormalTok{),}
    \AttributeTok{comp\_yr4 =} \FunctionTok{sum}\NormalTok{(comp\_yr4, }\AttributeTok{na.rm=}\ConstantTok{TRUE}\NormalTok{),}
    \AttributeTok{comp\_yr5 =} \FunctionTok{sum}\NormalTok{(comp\_yr5, }\AttributeTok{na.rm=}\ConstantTok{TRUE}\NormalTok{)}
\NormalTok{  )}

\NormalTok{housing\_sums\_boro }\OtherTok{\textless{}{-}}\NormalTok{   housing\_agg\_df\_boro\_grpd }\SpecialCharTok{\%\textgreater{}\%}
  \FunctionTok{mutate}\NormalTok{(}\AttributeTok{housingsum =}\NormalTok{ comp\_yr1 }\SpecialCharTok{+}\NormalTok{ comp\_yr2 }\SpecialCharTok{+}\NormalTok{ comp\_yr3 }\SpecialCharTok{+}\NormalTok{ comp\_yr4 }\SpecialCharTok{+}\NormalTok{ comp\_yr5,}
    \AttributeTok{yoy\_yr2 =} \FunctionTok{ifelse}\NormalTok{(comp\_yr1 }\SpecialCharTok{==} \DecValTok{0}\NormalTok{, }\ConstantTok{NA}\NormalTok{, (comp\_yr2 }\SpecialCharTok{{-}}\NormalTok{ comp\_yr1) }\SpecialCharTok{/}\NormalTok{ comp\_yr1 }\SpecialCharTok{*} \DecValTok{100}\NormalTok{),}
    \AttributeTok{yoy\_yr3 =} \FunctionTok{ifelse}\NormalTok{(comp\_yr2 }\SpecialCharTok{==} \DecValTok{0}\NormalTok{, }\ConstantTok{NA}\NormalTok{, (comp\_yr3 }\SpecialCharTok{{-}}\NormalTok{ comp\_yr2) }\SpecialCharTok{/}\NormalTok{ comp\_yr2 }\SpecialCharTok{*} \DecValTok{100}\NormalTok{),}
    \AttributeTok{yoy\_yr4 =} \FunctionTok{ifelse}\NormalTok{(comp\_yr3 }\SpecialCharTok{==} \DecValTok{0}\NormalTok{, }\ConstantTok{NA}\NormalTok{, (comp\_yr4 }\SpecialCharTok{{-}}\NormalTok{ comp\_yr3) }\SpecialCharTok{/}\NormalTok{ comp\_yr3 }\SpecialCharTok{*} \DecValTok{100}\NormalTok{),}
    \AttributeTok{yoy\_yr5 =} \FunctionTok{ifelse}\NormalTok{(comp\_yr4 }\SpecialCharTok{==} \DecValTok{0}\NormalTok{, }\ConstantTok{NA}\NormalTok{, (comp\_yr5 }\SpecialCharTok{{-}}\NormalTok{ comp\_yr4) }\SpecialCharTok{/}\NormalTok{ comp\_yr4 }\SpecialCharTok{*} \DecValTok{100}\NormalTok{))}

\NormalTok{housing\_sums\_boro}\SpecialCharTok{$}\NormalTok{avg\_yoy\_change }\OtherTok{\textless{}{-}} \FunctionTok{rowMeans}\NormalTok{(housing\_sums\_boro[, }\FunctionTok{c}\NormalTok{(}\StringTok{\textquotesingle{}yoy\_yr2\textquotesingle{}}\NormalTok{, }\StringTok{\textquotesingle{}yoy\_yr3\textquotesingle{}}\NormalTok{, }\StringTok{\textquotesingle{}yoy\_yr4\textquotesingle{}}\NormalTok{, }\StringTok{\textquotesingle{}yoy\_yr5\textquotesingle{}}\NormalTok{)])}

\NormalTok{housing\_final\_boro }\OtherTok{\textless{}{-}}\NormalTok{ housing\_sums\_boro[,}\FunctionTok{c}\NormalTok{(}\StringTok{"syringe\_yr"}\NormalTok{,}\StringTok{"borough"}\NormalTok{,}\StringTok{"housingsum"}\NormalTok{,}\StringTok{"avg\_yoy\_change"}\NormalTok{)] }\SpecialCharTok{\%\textgreater{}\%}
\NormalTok{                dplyr}\SpecialCharTok{::}\FunctionTok{rename}\NormalTok{(}\StringTok{"year"}\OtherTok{=}\StringTok{"syringe\_yr"}\NormalTok{)}
\end{Highlighting}
\end{Shaded}

\textbf{Joining Aggregated Data Sets Together for Borough Level
Geography}

\begin{Shaded}
\begin{Highlighting}[]
\CommentTok{\# first join}
\NormalTok{ syringe\_homeless\_boro}\OtherTok{\textless{}{-}} \FunctionTok{merge}\NormalTok{(syringe\_grouped\_boro,hmls\_final\_boro, }\AttributeTok{by =} \FunctionTok{c}\NormalTok{(}\StringTok{\textquotesingle{}year\textquotesingle{}}\NormalTok{,}\StringTok{"borough"}\NormalTok{), }\AttributeTok{all =} \ConstantTok{TRUE}\NormalTok{)}

\CommentTok{\# checking data types }
\NormalTok{syringe\_homeless\_boro}\SpecialCharTok{$}\NormalTok{year }\OtherTok{\textless{}{-}} \FunctionTok{as.integer}\NormalTok{(syringe\_homeless\_boro}\SpecialCharTok{$}\NormalTok{year)}
\NormalTok{housing\_final\_boro}\SpecialCharTok{$}\NormalTok{year }\OtherTok{\textless{}{-}} \FunctionTok{as.integer}\NormalTok{(housing\_final\_boro}\SpecialCharTok{$}\NormalTok{year)}

\CommentTok{\#second join}
\NormalTok{housing\_syringe\_df\_boro }\OtherTok{\textless{}{-}} \FunctionTok{merge}\NormalTok{(syringe\_homeless\_boro, }
\NormalTok{                           housing\_final\_boro, }
                           \AttributeTok{by=} \FunctionTok{c}\NormalTok{(}\StringTok{"year"}\NormalTok{,}\StringTok{"borough"}\NormalTok{),}\AttributeTok{all=}\ConstantTok{TRUE}\NormalTok{)}

\CommentTok{\#To keep more data points with the Agg, making the NA syrings values zero}
\NormalTok{housing\_syringe\_df\_boro }\OtherTok{\textless{}{-}}\NormalTok{ housing\_syringe\_df\_boro }\SpecialCharTok{\%\textgreater{}\%}
    \FunctionTok{mutate}\NormalTok{(}\AttributeTok{total\_syringes =} \FunctionTok{ifelse}\NormalTok{(}\FunctionTok{is.na}\NormalTok{(total\_syringes), }\DecValTok{0}\NormalTok{, total\_syringes))}

\CommentTok{\# Omitting the remaining nulls}
\NormalTok{data\_final\_boro\_nonulls}\OtherTok{\textless{}{-}} \FunctionTok{na.omit}\NormalTok{(housing\_syringe\_df\_boro)}
\end{Highlighting}
\end{Shaded}

Not checking the distributions of raw variables again, as I did before.
We know some have floors around zero, so I will just log the variables
again.

\paragraph{QUESTION 1: Does the Number of Housing Related Construction
Projects for the Previous 5 Years influence the Size of Adult Homeless
Population?}\label{question-1-does-the-number-of-housing-related-construction-projects-for-the-previous-5-years-influence-the-size-of-adult-homeless-population-1}

\textbf{MODEL 4: Relationship Between Logged Adult Homeless Population
and Logged Total Number of Completed Housing Related Construction
Projects for Preceding 5 Years at the Borough Level}

\begin{verbatim}
## 
## Call:
## lm(formula = log(avg_homeless_count + 1) ~ log(housingsum + 1), 
##     data = data_final_boro_nonulls)
## 
## Residuals:
##     Min      1Q  Median      3Q     Max 
## -3.9643 -1.3569 -0.2316  1.7439  3.4821 
## 
## Coefficients:
##                     Estimate Std. Error t value Pr(>|t|)    
## (Intercept)         -22.7823     3.9189  -5.813 3.99e-06 ***
## log(housingsum + 1)   2.3384     0.3101   7.541 5.27e-08 ***
## ---
## Signif. codes:  0 '***' 0.001 '**' 0.01 '*' 0.05 '.' 0.1 ' ' 1
## 
## Residual standard error: 1.9 on 26 degrees of freedom
## Multiple R-squared:  0.6862, Adjusted R-squared:  0.6741 
## F-statistic: 56.86 on 1 and 26 DF,  p-value: 5.265e-08
\end{verbatim}

\pandocbounded{\includegraphics[keepaspectratio]{FinalProject_JohnFerrara_files/figure-latex/m4_plot-1.pdf}}

Model 4 \textbf{is} Statisitcally Significant, checking validity.

\textbf{Model 4 Linearity Check}

\begin{Shaded}
\begin{Highlighting}[]
\CommentTok{\# Linearity Check}
\FunctionTok{ggplot}\NormalTok{(m4, }\FunctionTok{aes}\NormalTok{(}\AttributeTok{x=}\NormalTok{.fitted, }\AttributeTok{y=}\NormalTok{.resid)) }\SpecialCharTok{+} 
  \FunctionTok{geom\_point}\NormalTok{()}\SpecialCharTok{+}
  \FunctionTok{geom\_hline}\NormalTok{(}\AttributeTok{yintercept =} \DecValTok{0}\NormalTok{, }\AttributeTok{linetype =} \StringTok{"dashed"}\NormalTok{) }\SpecialCharTok{+}
  \FunctionTok{labs}\NormalTok{(}\AttributeTok{x =} \StringTok{"Fitted values"}\NormalTok{,}\AttributeTok{y =}\StringTok{"Residuals"}\NormalTok{)}
\end{Highlighting}
\end{Shaded}

\pandocbounded{\includegraphics[keepaspectratio]{FinalProject_JohnFerrara_files/figure-latex/m4_check1-1.pdf}}
Linearity Check doesnt seem to pass. There is a pattern in the portions
of the data.

\textbf{Model 4 Residual Distribution Check 1 }

\begin{Shaded}
\begin{Highlighting}[]
\CommentTok{\# Residual Dist.}
\FunctionTok{ggplot}\NormalTok{(}\AttributeTok{data =}\NormalTok{ m4, }\FunctionTok{aes}\NormalTok{(}\AttributeTok{x =}\NormalTok{ .resid)) }\SpecialCharTok{+}\FunctionTok{geom\_histogram}\NormalTok{(}\AttributeTok{binwidth =} \DecValTok{1}\NormalTok{) }\SpecialCharTok{+} \FunctionTok{xlab}\NormalTok{(}\StringTok{"Residuals"}\NormalTok{)}
\end{Highlighting}
\end{Shaded}

\pandocbounded{\includegraphics[keepaspectratio]{FinalProject_JohnFerrara_files/figure-latex/m4_check2-1.pdf}}

\textbf{Model 4 Residual Distribution Check 2}

\begin{Shaded}
\begin{Highlighting}[]
\CommentTok{\#Variablity of Constant}
\FunctionTok{ggplot}\NormalTok{(}\AttributeTok{data =}\NormalTok{ m4, }\FunctionTok{aes}\NormalTok{(}\AttributeTok{sample =}\NormalTok{ .resid)) }\SpecialCharTok{+}\FunctionTok{stat\_qq}\NormalTok{()}
\end{Highlighting}
\end{Shaded}

\pandocbounded{\includegraphics[keepaspectratio]{FinalProject_JohnFerrara_files/figure-latex/m4_check3-1.pdf}}

MODEL 4 may not be valid. Redsiduals and Predicted values have a pattern
in the points. The residuals have mostly normal distribution. Slight
deviation, but normal.

\textbf{MODEL 5: Relationship Between Logged Adult Homeless Population
and Logged Average YOY Percent Change in Completed Housing Related
Construction Projects for Preceding 5 Years at the Borough Level}

\begin{verbatim}
## 
## Call:
## lm(formula = log(avg_homeless_count + 1) ~ log(avg_yoy_change + 
##     1), data = data_final_boro_nonulls)
## 
## Residuals:
##     Min      1Q  Median      3Q     Max 
## -9.6693 -0.5597  1.0825  1.8315  2.3269 
## 
## Coefficients:
##                         Estimate Std. Error t value Pr(>|t|)    
## (Intercept)              10.9937     2.0885   5.264 4.42e-05 ***
## log(avg_yoy_change + 1)  -1.3169     0.6193  -2.126   0.0468 *  
## ---
## Signif. codes:  0 '***' 0.001 '**' 0.01 '*' 0.05 '.' 0.1 ' ' 1
## 
## Residual standard error: 3.122 on 19 degrees of freedom
##   (7 observations deleted due to missingness)
## Multiple R-squared:  0.1922, Adjusted R-squared:  0.1497 
## F-statistic: 4.521 on 1 and 19 DF,  p-value: 0.04681
\end{verbatim}

\pandocbounded{\includegraphics[keepaspectratio]{FinalProject_JohnFerrara_files/figure-latex/m5_plot-1.pdf}}
Model 5 \textbf{is} Statistic Significant, checking validity.

\textbf{Model 5 Linearity Check}

\begin{Shaded}
\begin{Highlighting}[]
\CommentTok{\# Linearity Check}
\FunctionTok{ggplot}\NormalTok{(m5, }\FunctionTok{aes}\NormalTok{(}\AttributeTok{x=}\NormalTok{.fitted, }\AttributeTok{y=}\NormalTok{.resid)) }\SpecialCharTok{+} 
  \FunctionTok{geom\_point}\NormalTok{()}\SpecialCharTok{+}
  \FunctionTok{geom\_hline}\NormalTok{(}\AttributeTok{yintercept =} \DecValTok{0}\NormalTok{, }\AttributeTok{linetype =} \StringTok{"dashed"}\NormalTok{) }\SpecialCharTok{+}
  \FunctionTok{labs}\NormalTok{(}\AttributeTok{x =} \StringTok{"Fitted values"}\NormalTok{,}\AttributeTok{y =}\StringTok{"Residuals"}\NormalTok{)}
\end{Highlighting}
\end{Shaded}

\pandocbounded{\includegraphics[keepaspectratio]{FinalProject_JohnFerrara_files/figure-latex/m5_check1-1.pdf}}
Slight pattern with outliers. May skew regression.

\textbf{Model 5 Residual Distribution Check 1 }

\begin{Shaded}
\begin{Highlighting}[]
\CommentTok{\# Residual Dist.}
\FunctionTok{ggplot}\NormalTok{(}\AttributeTok{data =}\NormalTok{ m5, }\FunctionTok{aes}\NormalTok{(}\AttributeTok{x =}\NormalTok{ .resid)) }\SpecialCharTok{+}\FunctionTok{geom\_histogram}\NormalTok{(}\AttributeTok{binwidth =} \DecValTok{4}\NormalTok{) }\SpecialCharTok{+} \FunctionTok{xlab}\NormalTok{(}\StringTok{"Residuals"}\NormalTok{)}
\end{Highlighting}
\end{Shaded}

\pandocbounded{\includegraphics[keepaspectratio]{FinalProject_JohnFerrara_files/figure-latex/m5_check2-1.pdf}}

\textbf{Model 5 Residual Distribution Check 2}

\begin{Shaded}
\begin{Highlighting}[]
\CommentTok{\#Variablitiy of Constant}
\FunctionTok{ggplot}\NormalTok{(}\AttributeTok{data =}\NormalTok{ m5, }\FunctionTok{aes}\NormalTok{(}\AttributeTok{sample =}\NormalTok{ .resid)) }\SpecialCharTok{+}\FunctionTok{stat\_qq}\NormalTok{()}
\end{Highlighting}
\end{Shaded}

\pandocbounded{\includegraphics[keepaspectratio]{FinalProject_JohnFerrara_files/figure-latex/m5_check3-1.pdf}}

Model 5 may not be Valid. Residuals are mostly normal, but outliers may
skew.

\paragraph{QUESTION 2: Does the size of the Adult Homeless Population
influence the Number of Used Syringes recovered by NYC
Parks?}\label{question-2-does-the-size-of-the-adult-homeless-population-influence-the-number-of-used-syringes-recovered-by-nyc-parks-1}

\textbf{MODEL 6: Relationship Between Logged Adult Homeless Population
and Logged Total Number of Used Syringes Collected at the Borough Level}

\begin{verbatim}
## 
## Call:
## lm(formula = log(total_syringes + 1) ~ log(avg_homeless_count + 
##     1), data = data_final_boro_nonulls)
## 
## Residuals:
##    Min     1Q Median     3Q    Max 
## -7.697 -2.024  2.045  3.827  4.218 
## 
## Coefficients:
##                             Estimate Std. Error t value Pr(>|t|)  
## (Intercept)                   1.6974     1.8896   0.898    0.377  
## log(avg_homeless_count + 1)   0.6889     0.2552   2.699    0.012 *
## ---
## Signif. codes:  0 '***' 0.001 '**' 0.01 '*' 0.05 '.' 0.1 ' ' 1
## 
## Residual standard error: 4.414 on 26 degrees of freedom
## Multiple R-squared:  0.2189, Adjusted R-squared:  0.1889 
## F-statistic: 7.287 on 1 and 26 DF,  p-value: 0.01205
\end{verbatim}

\pandocbounded{\includegraphics[keepaspectratio]{FinalProject_JohnFerrara_files/figure-latex/m6_plot-1.pdf}}

Model 6 \textbf{is} Statistically Significant. Checking Validity

\textbf{Model 6 Linearity Check}

\begin{Shaded}
\begin{Highlighting}[]
\CommentTok{\# Linearity Check}
\FunctionTok{ggplot}\NormalTok{(m6, }\FunctionTok{aes}\NormalTok{(}\AttributeTok{x=}\NormalTok{.fitted, }\AttributeTok{y=}\NormalTok{.resid)) }\SpecialCharTok{+} 
  \FunctionTok{geom\_point}\NormalTok{()}\SpecialCharTok{+}
  \FunctionTok{geom\_hline}\NormalTok{(}\AttributeTok{yintercept =} \DecValTok{0}\NormalTok{, }\AttributeTok{linetype =} \StringTok{"dashed"}\NormalTok{) }\SpecialCharTok{+}
  \FunctionTok{labs}\NormalTok{(}\AttributeTok{x =} \StringTok{"Fitted values"}\NormalTok{,}\AttributeTok{y =}\StringTok{"Residuals"}\NormalTok{)}
\end{Highlighting}
\end{Shaded}

\pandocbounded{\includegraphics[keepaspectratio]{FinalProject_JohnFerrara_files/figure-latex/m6_check1-1.pdf}}

\begin{Shaded}
\begin{Highlighting}[]
\DocumentationTok{\#\#}
\end{Highlighting}
\end{Shaded}

May not be valid, the poitns are clustered and not random.

\textbf{Model 6 Residual Distribution Check 1 }

\begin{Shaded}
\begin{Highlighting}[]
\CommentTok{\# Residual Dist.}
\FunctionTok{ggplot}\NormalTok{(}\AttributeTok{data =}\NormalTok{ m6, }\FunctionTok{aes}\NormalTok{(}\AttributeTok{x =}\NormalTok{ .resid)) }\SpecialCharTok{+}\FunctionTok{geom\_histogram}\NormalTok{(}\AttributeTok{binwidth =}\DecValTok{5}\NormalTok{) }\SpecialCharTok{+} \FunctionTok{xlab}\NormalTok{(}\StringTok{"Residuals"}\NormalTok{)}
\end{Highlighting}
\end{Shaded}

\pandocbounded{\includegraphics[keepaspectratio]{FinalProject_JohnFerrara_files/figure-latex/m6_check2-1.pdf}}

\textbf{Model 6 Residual Distribution Check 2}

\begin{Shaded}
\begin{Highlighting}[]
\CommentTok{\#Variablitiy of Constant}
\FunctionTok{ggplot}\NormalTok{(}\AttributeTok{data =}\NormalTok{ m6, }\FunctionTok{aes}\NormalTok{(}\AttributeTok{sample =}\NormalTok{ .resid)) }\SpecialCharTok{+} \FunctionTok{stat\_qq}\NormalTok{()}
\end{Highlighting}
\end{Shaded}

\pandocbounded{\includegraphics[keepaspectratio]{FinalProject_JohnFerrara_files/figure-latex/m6_check3-1.pdf}}

Model 6 may not be Valid. Residuals mostly normal.

\subsection{Results \& Conclusion}\label{results-conclusion}

Of the 6 different attempts at linear regressions to identify strong
relationships between varuables, only 5 had statistically significant
results. These results can be seen in the table below. For each of these
models, the variable was logged in order to get a more normal
distribution.

\textbf{RESULTS TABLE: Statistically Significant Models}

\begin{longtable}[]{@{}
  >{\raggedright\arraybackslash}p{(\linewidth - 16\tabcolsep) * \real{0.2446}}
  >{\raggedright\arraybackslash}p{(\linewidth - 16\tabcolsep) * \real{0.0944}}
  >{\raggedright\arraybackslash}p{(\linewidth - 16\tabcolsep) * \real{0.1245}}
  >{\raggedright\arraybackslash}p{(\linewidth - 16\tabcolsep) * \real{0.1674}}
  >{\raggedright\arraybackslash}p{(\linewidth - 16\tabcolsep) * \real{0.0472}}
  >{\raggedright\arraybackslash}p{(\linewidth - 16\tabcolsep) * \real{0.0858}}
  >{\raggedright\arraybackslash}p{(\linewidth - 16\tabcolsep) * \real{0.0773}}
  >{\raggedright\arraybackslash}p{(\linewidth - 16\tabcolsep) * \real{0.0773}}
  >{\raggedright\arraybackslash}p{(\linewidth - 16\tabcolsep) * \real{0.0815}}@{}}
\toprule\noalign{}
\begin{minipage}[b]{\linewidth}\raggedright
Regression Analysis
\end{minipage} & \begin{minipage}[b]{\linewidth}\raggedright
Geography Level
\end{minipage} & \begin{minipage}[b]{\linewidth}\raggedright
Dependent Variable
\end{minipage} & \begin{minipage}[b]{\linewidth}\raggedright
Independent Variable
\end{minipage} & \begin{minipage}[b]{\linewidth}\raggedright
R-squared
\end{minipage} & \begin{minipage}[b]{\linewidth}\raggedright
Adjusted R-squared
\end{minipage} & \begin{minipage}[b]{\linewidth}\raggedright
P-value
\end{minipage} & \begin{minipage}[b]{\linewidth}\raggedright
Correlation Type
\end{minipage} & \begin{minipage}[b]{\linewidth}\raggedright
Validity
\end{minipage} \\
\midrule\noalign{}
\endhead
\bottomrule\noalign{}
\endlastfoot
Adult Homeless Pop. \& Total Housing Proj. Prev 5 Yrs. (Logged) &
Community District & log(avg\_homeless\_count + 1) & log(housingsum + 1)
& 0.2725 & 0.2676 & 4.54e-12 & Direct & May not be valid \\
Adult Homeless Pop. \& Total Recovered Syringe Count (Logged) &
Community District & log(total\_syringe\_ests + 1) &
log(avg\_homeless\_count + 1) & 0.1096 & 0.1037 & 2.914e-05 & Direct &
Should be Valid \\
Adult Homeless Pop. \& Total Housing Proj. Prev 5 Yrs. (Logged) &
Borough & log(avg\_homeless\_count + 1) & log(housingsum + 1) & 0.6862 &
0.6741 & 5.27e-08 & Direct & May not be valid \\
Adult Homeless Pop. \& Avg. YOY Pct. Chg. Housing Proj. Prev 5 Yrs.
(Logged) & Borough & log(avg\_homeless\_count + 1) &
log(avg\_yoy\_change + 1) & 0.1922 & 0.1497 & 0.04681 & Indirect & May
not be valid \\
Adult Homeless Pop. \& Total Recovered Syringe Count (Logged) & Borough
& log(total\_syringes + 1) & log(avg\_homeless\_count + 1) & 0.2189 &
0.1889 & 0.01205 & Direct & May not be valid \\
\end{longtable}

\paragraph{Community District Level
Results}\label{community-district-level-results}

For the analysis at this Geographic level there was two models with a
p-value indicating statistical significance. The first was the
relationship between the Average Annual Adult Homeless and the total
number of Completed Housing related construction projects. About 27\% of
the variance in the Homeless population can be explained by the
Completed Construction Projects. The correlation was direct, which was
unexpected. Essentially, the model showed that as the number of
completed housing related construction projects for the previous 5 years
increased, so did the homeless population. This was surprising because
one would expect a negative sloped correlation, or that more
construction projects completed the less homeless people. These results
refute the Null hypothesis that there is no relationship between the
variables, and confirms the alternative hypothesis of the independent
variable, completed housing construction projects, impacting the
homeless population. However, as mentioned, the result was opposite of
what one may expect with a larger homeless population correlating with
more completed housing projects in the preceding 5 year window.

The second model that was statistically significant at the Community
District level, was the direct relationship between the number of
syringes collected in NYC Parks and from public disposal sites and the
size of the adult homeless population. Roughly 10\% of the variance in
Syringe Collection counts can be explained by the size of the adult
homeless population. These results reject the null hypothesis and
confirm the alternative hypothesis. A reduction in the homeless
population does reduce the number of syringes recovered.

\paragraph{Borough Level Results}\label{borough-level-results}

Due to various limitations of the data when examined at the community
district level, the same analysis was carried out at the borough level.
This generalization of geographic boundaries can help control for the
wide variance of city zoning for shelters, the varying presence of
parks, and uneven concentration of housing developments between
community districts. All three models at the borough level were
statistically significant.

Similar to the Community District analysis, a direct relationship was
found between the Adult homeless population and the Total Completed
Housing Construction Projects for the previous 5 year window. At the
borough level, about 67\% of the variance for the size of the homeless
population can be explained by the number of completed housing related
construction projects for the preceding 5 year window. This is a
stronger direct correlation than at the community district level.

Unlike the community district level analysis, the relationship between
the YOY average change in completed housing related construction
projects and the adult homeless population was statistically
significant. This relationship was found to be an indirect one, with
about 15\% of the variance in the homeless population being explained by
the Average YOY Change in projects for the preceding 5 year window. This
is more inline with what one would expect, with an increase in housing
related construction projects the homeless population would decrease.

Both of these models refute the null hypothesis of no relationship
between the completion of housing related construction project and the
adult homeless populations, but they do so in different ways.

Lastly, the model at the borough level that explores the relationship
between the syringes collected and the size of the adult homeless
population shows a direct relationship. An increase in the adult
homeless population does increase the number of syringes collected.
About 19\% of the variance in the amount of syringes collected can be
explained by the adult homeless population. This refutes the null
hypothesis and confirms the alternative hypothesis for question two.

\paragraph{Overall Conclusion}\label{overall-conclusion}

Overall, the null hypotheses were refuted for both questions, with both
alternative hypotheses being confirmed. The number of completed housing
related construction projects in the preceding 5 years for an area does
in fact correlate to the size of homeless adult shelter population.
Similarly, the size of the adult homeless population does correlate with
the number of syringes collected from parks and other public disposal
units.

The direct relationships between total construction projects completed
and homeless population data may be explained by wealthier and
gentrifying parts of the city consistently have the attention and
resources of developers to create more market-rate and luxury housing
via these construction project. The proliferation of such projects may
exacerbate the issues surrounding affordable housing. In my view, this
is confirmed by the the indirect relationship between the five year
average YOY percent change in construction project and the homeless
population. Larger YOY increases in completed construction projects may
imply a sudden increase in completed construction jobs in areas
typically neglected by high-end real estate developers and development
general, thus having an indirect relationship with the homeless
population. A large average YOY increase in completed construction
projects may also imply city planning aimed at helping areas in need of
more housing, again helping explain the indirect relationship.

Lastly, the direct relationship between the adult unhoused population
and the number of recovered syringes from parks and disposal sites does
suggest that this public health issue may be reduced in part by reducing
the adult homeless population by increasing the number of housing
construction projects.

\paragraph{Limitations of Analysis}\label{limitations-of-analysis}

While conclusions can be drawn from this analysis there are several data
limitations that should be noted. The housing data used in this analysis
is for housing-related completed construction projects, this data does
not differentiate between high-end developments nor between construction
projects that adds housing units to the market, or other types of
nuance. More granular housing data would be ideal for this type of
regression analysis, such as total units added to market, total
affordable housing units added to market, how many rent stabilized units
were removed from the market through renovation loopholes, etc.

Further limitations stem from the widely varying variables for each
geographic boundary. While Community District variation and the
geographic differences in the syringe collection data were accounted for
by the aggregation up to the borough level, the variance between
boroughs is not accounted for. Each of the boroughs zoning, parks land,
and shelter presence varies widely. There was simply not enough years of
data to analyze these variables at the city-level.

Additionally, the adult homeless population data used in this analysis
originated at counts from the shelter system. This does not take into
account the unhoused population that makes no use of the shelter system,
so there may be a large blind spot not considered in this analysis.

Lastly, several of the variables had floor effects on their
distributions due to the nature of the variables. Log transformations
were used, but there still were a large number of zero values for
several of the variables examined. This analysis, other than logging the
data, did not accommodate for this by removing outliers, weighing data
points, or by other means. With this said, as shown by the model
validity tests for linearity, some patterns are discernible in the data.
This may mean that these models are invalid and the conditions for
linearity may not be met. The inclusion of these zero values and some
outliers is another limitation of this analysis. Future work, should
attempt to further clean and process the data so as to create more valid
models.

\subsection{References}\label{references}

\textsuperscript{1} ``Home Prices, Wages, and Rent: Harvard Report
Reveals Housing Trends.'' NPR, 20 June 2024,
\href{}{https://www.npr.org/2024/06/20/nx-s1-5005972/home-prices-wages-paychecks-rent-housing-harvard-report\#:\textasciitilde:text=In\%20past\%20decades\%2C\%20it\%20was,Philly\%20is\%20giving\%20renters\%20cash}.
Accessed 7 Dec.~2024.

In addition to this citation, various internet sources like
StackOverFlow and other blogs were referenced for R syntax.

\end{document}
